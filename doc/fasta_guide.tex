\documentclass[11pt]{article}
\RequirePackage{helvet}
\newcommand{\CURRENT}{fasta-36.3.8i}
\renewcommand{\familydefault}{\sfdefault}
\usepackage{cite}
\usepackage{url}
\usepackage{fancyhdr}
\usepackage{url}
\usepackage{needspace}
\usepackage{longtable}
\addtolength{\oddsidemargin}{-0.75in}
\addtolength{\evensidemargin}{-0.75in}
\addtolength{\textwidth}{1.5in}
\addtolength{\topmargin}{-0.75in}
\addtolength{\textheight}{1.5in}

\lhead{\CURRENT}
\rhead{\today}
\cfoot{\thepage}
\pagestyle{fancy}
\newcommand{\FASTA}{\texttt{FASTA }}
\hyphenation{Swiss-Prot}

\parskip 0.5ex

\begin{document}
%% .he 'FASTA3.DOC''Release 3.6, March, 2011'

\section*{\Large{The FASTA program package}}

%% \begin{quote}
%% \emph{This document is undergoing extensive revision. 
%% Some parts of it are old; while still accurate, they are less
%% relevant.  Recent improvements to the FASTA programs are not well
%% documented, particularly with respect to options for selecting
%% databases and database sequences.}
%% \end{quote}

\section*{Introduction}

This documentation describes the version 36 of the FASTA program
package (see W. R. Pearson and D. J. Lipman (1988), ``Improved Tools
for Biological Sequence Analysis'', PNAS 85:2444-2448,\cite{wrp881}
W. R. Pearson (1996) ``Effective protein sequence comparison''
Meth. Enzymol. 266:227-258 \cite{wrp960}; and Pearson et. al. (1997)
Genomics 46:24-36 \cite{wrp973}.  Version 3 of the FASTA packages
contains many programs for searching DNA and protein databases and for
evaluating statistical significance from randomly shuffled sequences.

This document is divided into four sections: (1) A summary overview of
the programs in the FASTA3 package; (2) A guide to using the FASTA
programs; (3) A guide to installing the programs and
databases. Section (4) provides answers to some Frequently Asked
Questions (FAQs).  In addition to this document, the
\texttt{changes\_v36.html}, \texttt{changes\_v35.html} and
\texttt{changes\_v34.html} files list functional changes to the programs.
The \texttt{readme.v30..v36} files provide a more complete revision
history of the programs, including bug fixes.

The programs are easy to use; if you are using them on a machine that
is administered by someone else, you can focus on sections (1) and (2)
to learn how to use the programs.  If you are installing the programs
on your own machine, you will need to read section (3) carefully.

\emph{FASTA and BLAST} -- FASTA and BLAST have the same goal: to
identify statistically significant sequence similarity that can be
used to infer homology.  The FASTA programs offer several advantages
over BLAST:
\begin{enumerate}
\item
Rigorous algorithms unavailable in BLAST (Table \ref{table1}).  Smith-Waterman
(\texttt{ssearch36}), global: global (\texttt{ggsearch36}), and
global:local (\texttt{glsearch36}) programs are available, and these
programs can be used with \texttt{psiblast} PSSM profiles.
\item
Better translated alignments. \texttt{fastx36}, \texttt{fasty36},
\texttt{tfastx36}, and \texttt{tfastx36} allow frame-shifts in
alignments; frame-shifts are treated like gap-penalties, alignments
tend to be longer in error-prone reads.
\item
Better statistics. BLAST calculates very accurate statistics for
protein:protein alignments, but its model-based strategy is less
robust for translated-DNA:protein and DNA:DNA scores.  FASTA uses an
empirical estimation strategy, and now provides both search-based, and
high-scoring shuffle-based statistics (\texttt{-z 21}).
\item
More flexible library sequence formats.  The FASTA programs can read
FASTA, NCBI/ \texttt{formatdb}, and several other sequence formats, and can
directly query MySQL and Postgres databases. The programs offer
several strategies for specifying subsets of databases.
\item
A very efficient threaded implementation.  The FASTA programs are
fully threaded; both similarity scores and alignments can be
calculated in parallel on multi-core hardware.  On multi-core
machines, FASTA can be faster than BLAST while producing better
alignments with more accurate statistical estimates.
\item
  A powerful annotation facility.  The FASTA programs can incorporate
  functional site annotations, site variation, and domain-based
  sub-alignment scoring using annotations from sequence libraries.
  Scripts are available to download site and domain information from
  Uniprot, and domain information from Pfam and CATH.  Domain
  information can be used to sub-divide alignment scores to ensure
  that the aligned domain is homologous.

\end{enumerate}

In addition, the FASTA programs from \texttt{fasta-36.3.4} on provide
an option to produce very BLAST-like output (\texttt{-m BB}), so that
analysis pipelines require minimal modification.

\section{An overview of the \texttt{FASTA} programs}

\begin{table}
\caption{\label{table1} Comparison programs in the FASTA36 package}
\vspace{0.5ex}
\begin{tabular}{ p{0.8in} p{0.6in} p{4.6 in}}
\hline \\[-1.0ex]
FASTA \mbox{program} & BLAST equiv. & Description \\[1.2ex]
\hline \\[-1.0ex]
\texttt{fasta36} & \texttt{blastp}/ \texttt{blastn} &
Compare a protein sequence to a protein sequence 
database or a DNA sequence to a DNA sequence database using the FASTA 
algorithm \cite{wrp881,wrp960}.  Search speed and selectivity are 
controlled with the \emph{ktup}(wordsize) parameter.  For protein 
comparisons, \emph{ktup} = 2 by default; \emph{ktup} =1 is more sensitive 
but slower.  For DNA comparisons, \emph{ktup}=6 by default; \emph{ktup}=3 or 
\emph{ktup}=4 provides higher sensitivity.\\[1 ex]

\texttt{ssearch36} &  & Compare a protein sequence to a protein sequence 
database or a DNA sequence to a DNA sequence database using the 
Smith-Waterman algorithm \cite{wat815}. \texttt{ssearch36} uses SSE2
acceleration, and is only 2 - 5X slower than \texttt{fasta36} \cite{farrar2007}. \\[1 ex]

\texttt{ggsearch36}/ \texttt{glsearch36} &  & Compare a protein sequence to a protein sequence 
database or a DNA sequence to a DNA sequence database using
an optimal global:global (\texttt{ggsearch36}) or global:local
(\texttt{glsearch36}) algorithm.  \texttt{ggsearch36} limits
comparisons to sequences that are within 0.8 -- 1.25 the query sequence length,
 \textit{unless} the \texttt{-M low-high}  option is specified on
 the command line.\\[1 ex]

\texttt{fastx36}/ \texttt{fasty36} & \texttt{blastx} &
Compare a DNA sequence to a protein
sequence database, by comparing the translated DNA sequence in three
frames and allowing gaps and frameshifts.  \texttt{fastx36} uses a
simpler, faster algorithm for alignments that allows frameshifts only
between codons; \texttt{fasty36} is slower but can produce better alignments
because frameshifts are allowed within codons \cite{wrp971}.\\[1 ex]

\texttt{tfastx36}/ \texttt{tfasty36}& \texttt{tblastn} &
Compare a protein sequence to a DNA sequence
database, calculating similarities with frameshifts to the forward and
reverse orientations  \cite{wrp971}.\\[1 ex]

\texttt{fastf36/ tfastf36} &  &
Compares an ordered peptide mixture, as would be obtained by
Edman degradation of a CNBr cleavage of a protein, against a protein
(\texttt{fastf}) or DNA (\texttt{tfastf}) database \cite{wrp021}.\\[1 ex]

\texttt{fastm36/ tfastm36} &  &
Compares set of ordered short peptide fragments against a protein
(\texttt{fastm}) or DNA (\texttt{tfasts}) database
\cite{wrp021}. \texttt{fastm36} can also compare ordered
oligonucleotides against a DNA sequence database.\\[1 ex]

\texttt{fasts36/ tfasts36} &  &
Compares set of unordered short peptide fragments, as would be obtained
from mass-spec. analysis of a protein, against a
protein (\texttt{fasts}) or DNA (\texttt{tfasts}) database \cite{wrp021}.\\[1 ex]

\texttt{lalign36} & & Calculate multiple, non-intersecting alignments
using the sim2 implementation of the Waterman-Eggert
algorithm\cite{wat875} developed by Xiaoqui Huang and Web
Miller\cite{mil908}.  Statistical estimates are calculated from
Smith-Waterman scores of shuffled sequences. \\[1 ex]

\texttt{psisearch2} & \texttt{psiblast} & An improved version of \texttt{psiblast} with options to reduce alignment over-extension.\cite{wrp103,wrp171}\\[1 ex]

\hline \\
\end{tabular}
\end{table}

Although there are a large number of programs in this package, they
belong to three groups: (1) Traditional similarity searching programs:
\texttt{fasta36}, \texttt{fastx36}, \texttt{fasty36},
\texttt{tfastx36}, \texttt{tfasty36}, \texttt{ssearch36},
\texttt{ggsearch36}, and \texttt{glsearch36}; (2) Programs for
searching with short fragments: \texttt{fasts36}, \texttt{fastf36},
\texttt{tfasts36}, \texttt{tfastf36}, and \texttt{fastm36}; (3) A
program for finding non-overlapping local alignments: \texttt{lalign36}.
Programs that start with \texttt{fast} search protein databases, while
\texttt{tfast} programs search translated DNA databases.  Table \ref{table1}
gives a brief description of the programs.

In addition, there are several programs included. \texttt{map\_db} is
used to index FASTA format sequence databases for more efficient
scanning. \texttt{scripts/lav2plt.pl} can plot the \texttt{.lav}
files produced by \texttt{lalign -m 11} as postscript
(\texttt{lav2plt.pl --dev ps}) or SVG (\texttt{lav2plt.pl --dev svg}) output.

\section{Using the FASTA Package}
\subsection{Introduction/Overview}

All the FASTA sequence comparison programs use similar command line
options and arguments.  The simplest command line arguments are (in
order): the name of a query sequence file, a library file, and
(possibly) the \emph{ktup} parameter.  If command line options are
provided, they \emph{must} precede the standard query-file and
library-file arguments. Thus:
\begin{quote}
\texttt{fasta36 -s BP62 query.file library.file}
\end{quote}
will compare the sequences in \texttt{query.file} with those in
\texttt{library.file} using the \texttt{BLOSUM62} scoring matrix with
BLASTP gap penalties (\texttt{-11/-1}).

The program can also be run by typing:
\begin{quote}
\texttt{fasta36 -I}
\end{quote}
which presents the ``classic'' interative mode (this was
the default behavior before version \texttt{36.3.4}).
In interactive mode,
you will be prompted for: (1) the name of the test sequence file; (2)
the name of the library file; (3) whether you want ktup = 1 or 2. (
1 -- 6 for DNA sequences).

Current versions of the FASTA programs expect a query file and library, if you simply type ``\texttt{fasta36}'', you will see a short help message:
\begin{footnotesize}
\begin{quote}
\begin{verbatim}
% ssearch36
USAGE
 ssearch36 [-options] query_file library_file [ktup]
 ssearch36 -help for a complete option list

DESCRIPTION
 FASTA searches a protein or DNA sequence data bank
 version: 36.3.8h Aug, 2019

COMMON OPTIONS (options must preceed query_file library_file)
 -s:  scoring matrix;
 -f:  gap-open penalty;
 -g:  gap-extension penalty;
 -S   filter lowercase (seg) residues;
 -b:  high scores reported (limited by -E by default);
 -d:  number of alignments shown (limited by -E by default);
 -I   interactive mode;
\end{verbatim}
\end{quote}
\end{footnotesize}
``\texttt{fasta36 -help}'' (or any of the other program names in Table
I) provides complete listing of the options available for the program
and their default values.

The package includes several test files.  To check to make certain
that everything is working, you can try:
\begin{quote}
\begin{verbatim}
fasta36  ../seq/musplfm.aa ../seq/prot_test.lib
or
tfastx36 ../seq/mgstm1.aa ../seq/gst.nlib
\end{verbatim}
\end{quote}

\subsection{Sequence files}

The \texttt{fasta36} programs can read query and library files in many
standard formats (Section \ref{fastlibs}). The default file format for query and library files
-- the format that will be used if no additional file format
information is provided -- is \texttt{FASTA} format. Like
\texttt{BLAST}, version 36 can compare a query file with multiple
query sequences to a sequence database, performing an independent
search with each sequence in the query file.

FASTA format files consist of a description line, beginning
with a '$>$' character, followed by the sequence itself:
\begin{quote}
\begin{verbatim}
>sequence_name1 and description
A F A S Y T .... actual sequence.
F S S       .... second line of sequence.
>sequence_name2 and description 
PMILTYV ... sequence 2
\end{verbatim}
\end{quote}
All of the characters of the description line are read, and special
characters can be used to indicate additional information about the
sequence. In particular, a \texttt{'@:C 12345'} at the end of the
description line indicates that the first residue of the sequence has
coordinate \texttt{'12345'}, instead of starting at \texttt{'1'}.
Coordinates can be negative; a DNA sequence upstream from the start of
transcription could be displayed with negative coordinates.

In general, non-amino-acid/non-nucleotide sequences in the sequence
lines are ignored, with the exception of \texttt{'*'}, which indicates
a termination codon in a protein sequence, and can be used to indicate
the match to a termination codon in protein:DNA alignments.

FASTA format files from major sequence distributors, like the NCBI and
EBI, have specially formatted description lines, e.g.:\\
\indent
\texttt{
>np\_12345| example NCBI refseq sequence\\
}
or\\
\indent
\texttt{
>sp:gstm1\_human P01234 glutathione transferase GSTM1 - human\\
}
or
\indent
\texttt{
>sp|P09488|GSTM1\_HUMAN glutathione transferase GSTM1 - human\\
}

Several sample test files are included with the FASTA distribution:
\texttt{seq/*.aa} and \texttt{seq/*.seq}, as well as two small sequence
libraries, \texttt{seq/prot\_test.lib} and \texttt{seq/gst.nlib}.

You can build your own library by concatenating several sequence
files.  Just be sure that each sequence is preceded by a line
beginning with a '\texttt{>}' followed by a sequence name/description.  Sequences
entered with word processors should use a ``text'' mode, e.g. ``Save as
text'' with MS-WORD, with end of line characters and no special
formatting characters in the file.  The FASTA program cannot read
Microsoft Word .DOC files, or rich text (.RTF) files; query and
library sequence files should contain only sequence descriptions,
sequences, and end-of-line characters.

\subsection{Running the programs}
As mentioned earlier, the FASTA programs can be run either
interactively, by typing the name of a FASTA program (and possibly
command line options), followed by \texttt{-I} (\texttt{fasta36 -I})
or from the command line, entering command line options, and the
query and library file names. For searches of large databases that
may take several minutes (or longer), it is more convenient
to run searches from the command line, e.g.:
\begin{quote}
\begin{verbatim}
fasta36 query.file library.file > output.file
\end{verbatim}
\end{quote}
The command line shown above could be typed in a Unix or MacOSX
terminal window, or from the MS-Windows command line interface
(command.exe).  The command line syntax shown above works for all
the FASTA programs, e.g.:
\begin{quote}
\begin{verbatim}
lalign36 mchu.aa mchu.aa > mchu.laln
fastx36 mgstm1.seq prot_test.lseg > mgstm1.fx_out
ssearch36 mgstm1.aa xurtg.aa > mgstm1_xurtg.ss
\end{verbatim}
\end{quote}

\emph{Command line options} -- The FASTA programs provide a variety of
command line options that modify the default scoring matrix
(\texttt{-sBL62}) and gap penalties (\texttt{-f -11}, \texttt{-g -1}), other
algorithm parameters, the output options (\texttt{-E 0.1}, \texttt{-d 20},
\texttt{-m 9i}), and statistical procedures (\texttt{z -2}).  A complete
list of command line options is shown near the end of this document.
Unlike the \texttt{BLAST} programs, all \texttt{FASTA} command line options
must precede the query file name and library file name (and there are
no command line options available to specify the query and library
file names).  Thus, you should type:
\begin{quote}
\begin{verbatim}
ssearch36 -s BL62 -f -11 -g -1 query.file library.file > output.file
\end{verbatim}
\end{quote}
If you include \texttt{-I} as one of the options, you can provide
command line options (e.g. to change the scoring matrix or gap
penalties) without a query file or library file, and the program will
use the options but prompt for the necessary files .

\subsection{Interpreting the results}

Fig. \ref{ssearch_run} shows the output from a typical FASTA program
(\texttt{ssearch36}).  The output file can be
viewed as four parts: (a) the initial command line and description of
the query sequence used (mgstm1.aa, 218 aa) and library (PIR1, 13,143
entries); (b) a description of the search statistics, algorithm
(Smith-Waterman, SSE2 accelerated), and search parameters (BLOSUM50
matrix, gap penalties: -10 to open a gap, -2 for each residue in a
gap); (c) a list of high scoring library sequences, descriptions, similarity scores, and statistical significance; (d) the alignments that produced the scores.

\begin{figure}
\begin{footnotesize}
\begin{quote}
\begin{verbatim}
# ../bin/ssearch36 -q -w 80 ../seq/mgstm1.aa a
SSEARCH performs a Smith-Waterman search
 version 36.3.6 June, 2013(preload9)
Please cite:
 T. F. Smith and M. S. Waterman, (1981) J. Mol. Biol. 147:195-197; 
 W.R. Pearson (1991) Genomics 11:635-650
Query: ../seq/mgstm1.aa
  1>>>mGSTM1 mouse glutathione transferase M1 - 218 aa
Library: PIR1 Annotated (rel. 66) 
  5121825 residues in 13143 sequences

Statistics:  Expectation_n fit: rho(ln(x))= 7.4729+/-0.000484; mu= 2.0282+/- 0.027
 mean_var=56.9651+/-10.957, 0's: 9 Z-trim(119.4): 17  B-trim: 67 in 1/62
 Lambda= 0.169930
 statistics sampled from 13135 (13143) to 13135 sequences
Algorithm: Smith-Waterman (SSE2, Michael Farrar 2006) (7.2 Nov 2010)
Parameters: BL50 matrix (15:-5), open/ext: -10/-2
 Scan time:  3.820
The best scores are:                                                    s-w bits E(13143)
sp|P08010|GSTM2_RAT Glutathione S-transferase Mu 2; GST 4-4; GT  ( 218) 1248 312.0 7.7e-86
sp|P04906|GSTP1_RAT Glutathione S-transferase P; Chain 7; GST -  ( 210)  344 90.4 3.8e-19
sp|P00502|GSTA1_RAT Glutathione S-transferase alpha-1; GST 1-1   ( 222)  237 64.1 3.2e-11
sp|P14942|GSTA4_RAT Glutathione S-transferase alpha-4; GST 8-8   ( 222)  179 49.9 6.1e-07
sp|P12653|GSTF1_MAIZE Glutathione S-transferase 1; GST class-pi  ( 214)  120 35.4   0.013
sp|P04907|GSTF3_MAIZE Glutathione S-transferase 3; GST class-pi  ( 222)  115 34.2   0.032
sp|P20432|GSTT1_DROME Glutathione S-transferase 1-1; DDT-dehydr  ( 209)  100 30.5    0.38
sp|P11277|SPTB1_HUMAN Spectrin beta chain, erythrocytic; Beta-   (2137)  108 31.6     1.9
... (alignments deleted) ...
>>sp|P14942|GSTA4_RAT Glutathione S-transferase alpha-4; GST 8-8;             (222 aa)
 s-w opt: 179  Z-score: 231.0  bits: 49.9 E(13143): 6.1e-07
Smith-Waterman score: 179; 25.6% identity (54.5% similar) in 211 aa overlap (5-207:7-207)
                 10        20        30        40        50          60         70     
mGSTM    MPMILGYWNVRGLTHPIRMLLEYTDSSYDEKRYTMGDAPDFDRSQWLNEKF-KLG-LDFPNLPYL-IDGSHKITQSNA
             : :.. ::  . :: ::  .   ..:         .: ...   ::. : : : : ..: . :::   .::. :
sp|P14 MEVKPKLYYFQGRGRMESIRWLLATAGVEFEE---------EFLETREQYEKLQKDGCLLFGQVPLVEIDG-MLLTQTRA
               10        20        30                 40        50        60         70
          80        90       100       110       120       130       140            150
mGSTM  ILRYLARKHHLDGETEEERIRADIVENQVMDTRMQLIMLCYNPDFEKQKPEFLKTIPEKMKLYSEF--LGK---RPWFAG
       :: ::: :..: :.  .::.: :.  . ..:  :..:   ..   ::..   : .   : . .  :  . :   . ...:
sp|P14 ILSYLAAKYNLYGKDLKERVRIDMYADGTQDLMMMIIGAPFKAPQEKEESLALAVKRAKNRYFPVFEKILKDHGEAFLVG
               80        90       100       110       120       130       140       150
              160       170       180       190       200       210            
mGSTM  DKVTYVDFLAYDILDQYRMFEPKCLDAFPNLRDFLARFEGLKKISAYMKSSRYIATPIFSKMAHWSNK    
       ......:.   . . . .      :. :: :. : .:. ..  :. ... .     :               
sp|P14 NQLSWADIQLLEAILMVEEVSAPVLSDFPLLQAFKTRISNIPTIKKFLQPGSQRKPPPDGHYVDVVRTVLKF
              160       170       180       190       200       210       220  
... (alignments deleted) ...
218 residues in 1 query   sequences
5121825 residues in 13143 library sequences
 Tcomplib [36.3.6 May, 2013(preload9)] (4 proc in memory [0G])
 start: Thu Jun  6 11:23:28 2013 done: Thu Jun  6 11:23:30 2013
 Total Scan time:  3.820 Total Display time:  0.130
Function used was SSEARCH [36.3.6 May, 2013(preload9)]
\end{verbatim}
\end{quote}
\end{footnotesize}
\vspace{-4.0ex}

\caption{\label{ssearch_run}\texttt{ssearch36} results}
\vspace{1.0ex}
Comparison of \texttt{seq/mgstm1.aa} against a small protein database
(\texttt{pir1.lseg}). Some high-scoring sequences and all but one
alignment were removed to reduce the output size.
\end{figure}

\subsubsection{Identifying homologs}
In the description section (which starts: \texttt{The best scores
  are:}), four numbers after the description of each library sequence
are shown: (i) (in parentheses) the length of the library sequence;
(ii) the raw Smith-Waterman score for the alignment (\texttt{s-w}; for
the \texttt{fasta36}, \texttt{[t]fast[x,y]36} programs, this column
would be labeled \texttt{opt}, for the \emph{opt}imized -- banded
Smith-Waterman -- score), (iii) the \emph{bit} score, and (iv) the
expectation (E()), or statistical significance, of the alignment
score.  The E()-value depends on the size of the database searched, in
this case, 13,143 sequences, so the database size is given at the top
of the list.

The bit score is equivalent to a BLAST bit score; together with the
length of query and library sequences, it can be used to calculate the
significance of the alignment.\footnote{$E(D) = D m n 2^{-b}$,
where $D$ is the number of sequences in the database, $m, n$ are the
lengths of the two sequences, and $b$ is the bit score.}  Bit scores
are convenient because they provide a matrix independent score that
can be compared with other searches performed with other matrices and
gap penalties against other databases.  However, the E()-value, or
expectation, provides the most direct measure of the statistical
significance of the match.

In this example, the \texttt{GSTP1\_RAT}, \texttt{GSTA1\_RAT}, and
\texttt{GSTA4\_RAT} proteins share strong significant similarity
(better than $E() < 6.1 \times 10^{-7}$ ), while the
\texttt{GSTF1\_MAIZE}, \texttt{GSTF3\_MAIZE}, and
\texttt{GSTT1\_DROME} sequences do not share significant similarity
($E() < 0.001$).  However, \texttt{GSTF1\_MAIZE},
\texttt{GSTF3\_MAIZE}, and \texttt{GSTT1\_DROME} are all glutathione
transferase homologs, they simply do not share statistically
significant similarity with this particular \texttt{mGSTM1} query.
Statistically significant sequence similarity scores \emph{can} be
used to infer \emph{homology} (common ancestry), but non-significant
scores \emph{cannot} be used to infer \emph{non-homology}.

While percent identity is often used to characterize the quality of an
alignment and the likelihood that it reflects homology, the E()-value
is a much more reliable value for homology infernence (once homology is
established, the percent identity is much more useful for estimating
evolutionary distance).  Often sequences that share less than 30\%
identity will share very significant similarity (in the example above
\texttt{mgstm1.aa} and \texttt{GSTA4\_RAT}, with E() $<$ 6.1E-07 are
25.6\% identical).  The expectation value captures information about
conservative replacements, identities, and alignment length to provide
a \emph{single} value that captures the significance of the alignment.

For protein searches, library sequences with E()-values $<$ 0.001 for
searches of a 10,000 entry protein database are almost always
homologous. Some sequences with E()-values from 1 - 10 may also be
related, but unrelated sequences ( 1--10 per search) will have scores
in this range as well.

E()-values $<$ 0.001 can reliably be used to infer homology, assuming
that the statistical estimates are accurate.  The two most common
causes of statistical problems are low-complexity regions and
amino-acid composition bias.  Low-complexity regions are can be
identified using the \texttt{pseg} program \cite{woo935}, and filtered
out using the \texttt{-S} option. Composition bias rarely produces
highly-signficiant E()-values, but can cause unrelated sequences to
have E()-values between 0.01 and 0.001. The FASTA programs offer two
shuffle-based strategies for evaluating composition bias; calculating
similarity scores for random sequences with the same length and amino
acid composition (\texttt{-z 11} $..$ \texttt{16}),\footnote{Random
  shuffles are performed for pairwise alignments and \texttt{lalign36}
  by default.} and calculating statistical estimates derived from
shuffles of the high-scoring sequences (\texttt{-z 21, 22, 24, 25,
  26}).

When \texttt{-z 21 .. 26} shuffles are performed, the FASTA36 programs
present two E()-values in the list of high scoring sequences and the
alignments; the traditional one based on the library search, and a
second \texttt{E2()} value, based on the shuffles of the high scoring
sequences.  \texttt{-z 21 .. 26} shuffles are most useful for
evaluating the significance of translated-DNA:protein searches like
\texttt{fastx36}.  Out-of-frame translations can produce cryptic low
complexity regions, which are most apparent when the high-scoring
sequences are shuffled.  \texttt{-z 21} shuffles are more efficient
than individual sequence shuffles, because the set of high scoring
sequences is shuffled 500 -- 1,000 times, rather than 500 shuffles for
each of 50 -- 100 high scoring library sequences.  It is almost as
effective, because homologous sequences share similar amino-acid
composition.

The statistical routines assume that the library contains a large
sample of unrelated sequences.  If the library contains fewer than
500 sequences (\texttt{MAX\_RSTATS}), then the library sequences are
shuffled to produce 500 random scores, from which lambda and K
statistical parameters are estimated. If the library contains a large
number of \emph{related} sequences, then the statistical parameters
should be estimated by using the \texttt{-z 11-15}, options.
\texttt{-z} options greater than 10 calculate a shuffled similarity
score for each library sequence, in addition to the unshuffled score,
and estimate the statistical parameters from the scores of the
shuffled sequences.

\subsubsection{Looking at alignments}

The description section described above contains the critical
information for inferring homology, the \texttt{E()}-value.  The
alignment section shows the actual alignments that produced the
similarity score and statistical estimates.  In
Fig. \ref{ssearch_run}, the alignment display reports the percent
identity, percent similarity (number of aligned residues with BLOSUM50
values $\ge$ 0), and the boundaries of the alignment.  Note that for
the \texttt{ssearch36} and \texttt{fasta36}, the alignment shown can
include residues that are not part of the best local alignment
(e.g. residues 1--5 and 207--218 in \texttt{mGSTM1} in
Fig. \ref{ssearch_run}).  The amount of additional sequence context
shown is the alignment line length (60 residues, set by \texttt{-w
  len}) divided by 2 by default, but can be adjusted with the
\texttt{-W context} option.

\begin{figure}
\begin{footnotesize}
\begin{verbatim}
>>GST26_SCHMA Glutathione S-transferase class-mu  (218 aa)
 initn: 422 init1: 359 opt: 407  Z-score: 836.8  bits: 162.0 E(437847): 3.7e-39
Smith-Waterman score: 451; 42.4% identity (73.4% similar) in 203 aa overlap (6-208:6-203)

               10        20        30        40        50        60        70        80
mGSTM1 MPMILGYWNVRGLTHPIRMLLEYTDSSYDEKRYTMGDAPDFDRSQWLNEKFKLGLDFPNLPYLIDGSHKITQSNAILRYL
            :::.:.::..: :.:::. ...:.:. :   :  ..:   : :.::::::.:::::: :::. :.::: ::.::.
GST26_ MAPKFGYWKVKGLVQPTRlllehleetyeeRAY---DRNEIDA--WSNDKFKLGLEFPNLPYYIDGDFKLTQSMAIIRYI
               10        20        30           40          50        60        70     

               90       100       110       120       130       140       150       160
mGSTM1 ARKHHLDGETEEERIRADIVENQVMDTRMQLIMLCYNPDFEKQKPEFLKTIPEKMKLYSEFLGKRPWFAGDKVTYVDFLA
       : ::.. :   .:: . ...:. :.: :: .. ..:: ..:  : .::. .: ..:.... :... .. :. ::. ::. 
GST26_ ADKHNMLGACPKERAEISMLEGAVLDIRMGVLRIAYNKEYETLKVDFLNKLPGRLKMFEDRLSNKTYLNGNCVTHPDFML
          80        90       100       110       120       130       140       150     

              170       180       190       200       210             
mGSTM1 YDILDQYRMFEPKCLDAFPNLRDFLARFEGLKKISAYMKSSRYIATPIFSKMAHWSNK     
       :: ::    .. .::. ::.: .:   .: : .:. :..:::::  :.               
GST26_ YDALDVVLYMDSQCLNEFPKLVSFKKCIEDLPQIKNYLNSSRYIKWPLQGWDATFGGGDTPPK
         160       170       180       190       200       210        
\end{verbatim}
\end{footnotesize}

\vspace{-3.0ex}
\caption{\label{seg-aln}Alignment with \texttt{-S} filtered sequence}
\end{figure}

Fig. \ref{seg-aln} shows an example of a \texttt{fasta36} alignment
produced using the \texttt{-S} option to filter out lower-case (low
complexity) residues.  Here, additional scores (\texttt{initn},
\texttt{init1} are shown, in addition to the \texttt{opt} score which
is used to rank the sequences and calculate statistical significance.
The \texttt{init1} score is the highest scoring alignment without
gaps; \texttt{initn} is a score that combines consistent
(non-overlapping) runs without gaps, and \texttt{opt} is the score of
a banded Smith-Waterman of width 16 for \emph{ktup=2} that is applied
to sequences with \texttt{initn} scores over the optimization
threshold. In Fig. \ref{seg-aln},  the \texttt{init1} score is
based on the long, un-gapped region from residues 46--208 in
\texttt{mGSTM1}, while the \texttt{initn} and \texttt{opt} scores
include the other regions joined by gaps.  The \texttt{initn} score is
higher than the \texttt{opt} score, because it uses a simpler,
length-independent, gap penalty.

The \texttt{init1} and \texttt{initn} scores are shown for
historical reasons, and can be used to illustrate the FASTA algorithm.
But the \texttt{opt} score is the most reliable and sensitive score
for inferring homology; the others can be ignored.

For \texttt{fasta36} with proteins, the final alignment and score is
calculated with the Smith-Waterman algorithm. For DNA sequences, a
banded Smith-Waterman is used. (The \texttt{-A} option produces banded
Smith-Waterman alignments for proteins, and full Smith-Waterman for
DNA.)  In Fig. \ref{seg-aln}, the \texttt{opt} score and
\texttt{Smith-Waterman} scores are calculated on exactly the same
alignment, but the \texttt{opt} score excludes the contribution from
the ``low-complexity'' region between 19--30 in \texttt{GST26\_SCHMA}.

\subsubsection{Results without alignments}

While sequence alignments are very informative, it is often not
practical to examine all the statistically significant alignments in
large-scale searches. The \texttt{-m 9} and \texttt{-m 8} options
present summaries of each alignment (alignment boundaries, percent
identity, and other information) in a much more compact
form. \texttt{-m 9i} adds three columns to the summary report line,
the fraction identical, fraction similar, and alignment length (in
addition, variants are reported if they affect identity). \texttt{-m
  9I} is similar to \texttt{-m 9i}, but also reports domain content if
annotations are used.  \texttt{-m 9c} or \texttt{-m 9C} (see options
below) provide a detailed encoding of the alignment, that allows it to
be reconstructed. For large-scale searches, we routinely use
\texttt{-m 8} with the \texttt{-d 0} option, which sets the number of
alignments shown to 0 (thus none are shown).  Alternatively, the
\texttt{-m 8} and \texttt{-m 8C} ouput options produce BLAST-format
tabular results summaries (\texttt{-m 8C} provides commented tabular
results).  \texttt{-m 8CC} adds an alignment CIGAR string and
annotation string to the BLAST tabular format.  \texttt{-m BB}
produces an output that mimics BLAST output (with alignments).

\subsubsection{Alignments with annotations}

The command line \texttt{-V} option (described below) causes the FASTA
programs to ``decorate'' its sequence alignments with annotation
information, such as functional sites, variants, and domain-based
sub-alignment scores.  For example, a comparison of the
\texttt{seqs/gstm1\_human.vaa} sequence with SwissProt using the
\texttt{scripts/ann\_feats\_up\_www2.pl} script:
\begin{footnotesize}
\begin{quote}
\begin{verbatim}
ssearch36 -m 9i -V \!../scripts/ann_feats_up_www2.pl ../seq/gstm1\_human.vaa /slib/swissprot.fa
\end{verbatim}
\end{quote}
\end{footnotesize}
Produces the following addtional output:
\begin{footnotesize}
\begin{verbatim}
Annotation symbols:
 = : Active site
 * : Modified
 # : Substrate binding
 ^ : Metal binding
 @ : Site

The best scores are:                       s-w bits E(458668) %_id  %_sim  alen
sp|P09488.3|GSTM1_HUMAN Glutathione ( 218) 1500 375.2 2.4e-103 1.000 1.000  218 K173N;S210T;
sp|Q03013.3|GSTM4_HUMAN Glutathione ( 218) 1375 344.5 4.4e-94 0.904 0.963  218 S2P;A160V;L208V;Y209F;R211K;V212M;
sp|Q5R8E8.3|GSTM2_PONAB Glutathione ( 218) 1310 328.5 2.9e-89 0.862 0.959  218
sp|Q9TSM5.3|GSTM1_MACFA Glutathione ( 218) 1308 328.0   4e-89 0.858 0.954  218
sp|Q9TSM4.3|GSTM2_MACFA Glutathione ( 218) 1307 327.7 4.8e-89 0.862 0.954  218
sp|P28161.2|GSTM2_HUMAN Glutathione ( 218) 1306 327.5 5.7e-89 0.853 0.959  218 S173N;
sp|P46439.3|GSTM5_HUMAN Glutathione ( 218) 1305 327.2 6.7e-89 0.876 0.954  218 L179P;
\end{verbatim}
\end{footnotesize}
This summary of high scoring hits shows one of the effects of
annotation---substitution of variant residues to increase the score.
In this case, the query sequence \texttt{gstm1\_human.vaa} is a known
variant of the canonical \texttt{GSTM1\_HUMAN}/\texttt{P09488} UniProt
sequence.  Without the \texttt{-V} annotation option,
\texttt{gstm1\_human.vaa} would be 99\% identical to
\texttt{GSTM1\_HUMAN}, but because UniProt documents the variant
residues in the feature table, the \texttt{K173N} and \texttt{S210T}
substutions are made in the library (subject) sequence, producing a
perfect match.

In addition to the variant substitution shown above, the alignments
provide a more complete view of the annotations available on the
library (subject) proteins.  Below is the report for the
\texttt{GSTM4\_HUMAN} alignment:
\begin{footnotesize}
\begin{quote}
\begin{verbatim}
>>sp|Q03013.3|GSTM4_HUMAN Glutathione S-trans             (218 aa)
 Variant: 2P=2P : S2P : UniProtKB FT ID: VAR_033979
 Site:@ : 7Y=7Y : Site: Glutathione binding
 Site:@ : 46W=46W : Site: Glutathione binding
 Site:@ : 59N=59N : Site: Glutathione binding
 Site:@ : 72Q=72Q : Site: Glutathione binding
 Region: 2-88:2-88 : score=599; bits=150.1; Id=0.989; Q=415.2 :  GST N-terminal :1
 Site:# : 116Y=116Y : Substrate binding: Substrate
 Variant: 160V=160V : A160V : UniProtKB FT ID: VAR_033980
 Variant: 208V=208V : L208V : UniProtKB FT ID: VAR_049487
 Region: 90-208:90-208 : score=699; bits=175.1; Id=0.833; Q=489.3 :  GST C-terminal :2
 Variant: 209F=209F : Y209F : UniProtKB FT ID: VAR_049488
 Variant: 211K=211K : R211K : UniProtKB FT ID: VAR_049489
 Variant: 212M=212M : V212M : UniProtKB FT ID: VAR_049490
 s-w opt: 1375  Z-score: 1823.2  bits: 344.5 E(458668): 4.4e-94
Smith-Waterman score: 1375; 90.4% identity (96.3% similar) in 218 aa overlap (1-218:1-218)
\end{verbatim}
\end{quote}
\end{footnotesize}
This report can be broken into three parts: (1) information on
variants, described above, (2) information on annotated sites, and (3)
information on annotated domains.  For each annotated site, the
coordinate and amino-acid resdiue in the query and library (subject)
sequence is shown, as well as the conservation state (\texttt{=} in
all these examples).  For annotated domains, the overall alignment
score is broken into pieces, based on the boundaries of the domains.
In this case, the full alignment extends from residues 1--218,
producing a raw Smith-Waterman score of 1375 and a bit score of
344.5.  The GST N-terminal domain is annotated from residue 2--88 on
\texttt{GSTM4\_HUMAN}, and the 2--88 region of the alignment produces
a score of 599 and bit score of 150.1.  This region is 98.9\%
identical, and the probability of that similarity score is
$10^{-41.52}$ (the Qvalue score is $-10 log_{10} P$).  The
alignment associated with GST C-terminal domain is slightly less well
conserved (83.3\% identical), but longer, so it produces a higher
Smith-Waterman score (699), bit score (175.1) and Q-value.

Sub-alignment scores can be used to identify cases of alignment
over-extension \cite{wrp136}, where an alignment extends well beyond
the homologous domain.  In this case, the homologous region will
produce the vast majority of the score, and the non-homologous
over-extension will produce very little score.  For example, when
\texttt{SRC8\_HUMAN} aligns with \texttt{LASP1\_MOUSE}, the alignment
spans 200 residues, but only about 49 of those residues, an
\texttt{SH3} domain, are homologous:

\begin{footnotesize}
\begin{quote}
\begin{verbatim}
>>sp|Q61792.1|LASP1_MOUSE LIM and SH3 domain protein 1;  LASP-1;              (263 aa)
 Region: 369-398:66-95 : score=20; bits=13.7; Id=0.200; Q=0.0 :  Nebulin_repeat  InterPro
 Region: 400-434:97-131 : score=-8; bits=8.7; Id=0.150; Q=0.0 :  Nebulin_repeat  InterPro
 Region: 435-499:132-203 : score=13; bits=11.4; Id=0.197; Q=0.0 :  NODOM :0
 Region: 499-547:204-261 : score=124; bits=47.8; Id=0.474; Q=92.1 :  SH3  InterPro
 s-w opt: 148  Z-score: 253.4  bits: 55.6 E(459565): 1.2e-06
Smith-Waterman score: 159; 26.3% identity (55.6% similar) in 205 aa overlap (369-547:66-261)
\end{verbatim}
\end{quote}
\end{footnotesize}
The 150 aligned residues outside the \texttt{SH3} homology produce
less than 20\% of the alignment score, while spannign to
non-homologous Nebulin repeat domains.

\subsection{Program Options}

Command line options are available to change the scoring parameters
and output display. Unlike the NCBI BLAST programs, command line
options \emph{must} precede the query file name and library file name
arguments.  To see the command-line options for a program and their
defaults, type \texttt{program\_name -help}, e.g. \texttt{fasta36
  -help} or \texttt{ssearch36 -help}. For a quick list of the most
common options, just type the program name without any options
(e.g. \texttt{fasta36$<$ret$>$}).

\subsubsection{Command line options}
\begin{description}
\item[\texttt{-a}] (\texttt{fasta36}, \texttt{ssearch36},
  \texttt{glsearch36}, \texttt{fasts36}) show both sequences in their
  entirety.
\item[\texttt{-A}] force Smith-Waterman alignments for
  \texttt{fasta36} DNA sequences.  By default, only \texttt{fasta36}
  protein sequence comparisons use Smith-Waterman alignments.
  Likewise, for proteins, use band alignments (Smith-Waterman is used
  by default).
\item[\texttt{-b \#}] Number of sequence scores to be shown on output.
  In the absence of this option, \texttt{fasta36} (and
  \texttt{ssearch36}) display all library sequences obtaining
  similarity scores with expectations less than the expectation (-E)
  threshold, 10.0 for proteins, and 2.0 for DNA:DNA and
  protein:translated DNA.  The \texttt{-b \#} option can limit the
  display further.  There are two ``sub-modes'' of \texttt{-b}.
  \texttt{-b =100} will force 100 high scores to be displayed,
  regardless of the expectation (\texttt{-E}) threshold, and
  \texttt{-b >1} will show at least \texttt{1}, but is otherwise
  limited by \texttt{-E}.  Thus, \texttt{-b 10} will show \emph{no
    more than} 10 results, limited by \texttt{-E}; \texttt{-b =10}
  will always show \emph{exactly} \texttt{10} results, and \texttt{-b
    >5} will show \emph{at least} \texttt{5} results, but could show
  many more if more results have e\-values $\le$ \texttt{-E e\_cut}.
\item[\texttt{-c \#,\#}] (\texttt{fasta36}, \texttt{[t]fast[x,y]36}
  only) Fraction of alignments optimized (second value is fraction of
  sequences joined). FASTA36 uses a statistical threshold strategy
  that joins and optimizes only the fraction of the alignments with an
  \texttt{initn} score expected \texttt{-c} times.  Thus, \texttt{-c
    0.05} should optimize about 5\% of sequences.  The actual number
  of sequences optimized (and joined) is displayed in the scoring
  parameters line.  Thus:
\begin{quote}
\begin{verbatim}
Parameters: BL50 matrix (15:-5), open/ext: -10/-2
 ktup: 2, E-join: 1 (0.687), E-opt: 0.2 (0.294), width:  16
\end{verbatim}
\end{quote}
reports that 20\% of the sequences in the database should have been
band-optimized, and 29.4\% were. Reducing the \texttt{-c opt} fraction
improves performance, but dropping the fraction below 0.02 can
reduce the accuracy of the statistical estimates.

\texttt{-c O} (letter 'O') sets the joining/optimization
thresholds as they were prior to \texttt{fasta-36.3.3} (original thresholds). Positive
values set the thresholds to specific score values, as was the case
in older versions of \texttt{fasta}.

\item[\texttt{-C}]
length of the sequence name printed at the beginning of alignment
lines (default 6 characters).
\item[\texttt{-d \#}]
Maximum number of alignments to be displayed (must be \texttt{<=} to the number of descriptions, \texttt{-b \#})
\item[\texttt{-D}]
  Provide some debugging output.  Used in conjunction
  with the \texttt{-e expand\_script.sh}, the \texttt{link\_acc\_file}
  and \texttt{link\_lib\_file} are not deleted during the run; so that
  \texttt{expand\_script.sh} scripts can be tested.

\item[\texttt{-e expand\_script.sh}]

  Expand the set of sequences that
  are aligned to beyond the set of sequences searched. When the
  \texttt{-e expand\_script.sh} option is used, the
  \texttt{expand\_script.sh} script is run after the initial search
  scan but before the list of high-scoring sequences is displayed.
  \texttt{expand\_script.sh} is given a single argument, the name of a
  file that contains a list of accession strings (the text between the
  \texttt{>} and the first space ('\textvisiblespace') character
  followed by the E()-value for the sequence (separated by a
  \texttt{<tab>} character), e.g.:
\begin{quote}
\begin{verbatim}
sp|P08010|GSTM2_RAT<tab>2.69e-86
sp|P09211|GSTP1_HUMAN<tab>1.51e-20
sp|P04906|GSTP1_RAT<tab>1.16e-19
sp|P00502|GSTA1_RAT<tab>9.5e-12
\end{verbatim}
\end{quote}
The script should produce a fasta-formatted list of additional
sequences printed to \texttt{stdout}.  The script is run with the command:
\begin{quote}
\begin{verbatim}
expand_script.sh link_acc.tmp_file > link_lib.tmp_file
\end{verbatim}
\end{quote}
The sequences in \texttt{link\_lib.tmp\_file} (a temporary file name is
actually used, and the file is deleted unless the \texttt{-D} option
is used) are then compared and, if they are significant, included in
the list of high scoring sequences and the alignments.  The expanded
set of sequences does not change the database size or statisical
parameters, it simply expands the set of high-scoring sequences.

The \texttt{fasta36/misc} directory contains
\texttt{expand\_uniref50.pl} that uses a mySQL table based on the
\texttt{uniref50} clusters.  Using the script and the
\texttt{uniref50} cluster information, one can search
\texttt{uniref50.fasta}, but then expand the hits so that
\texttt{uniprot} appears to be searched.

\item[\texttt{-E e\_cut [e\_cut\_r]}] Limit the number of scores and
  alignments shown based on the expected number of scores.  Used to
  override the expectation value of 10.0 (protein:protein; 5.0
  translated-DNA:protein; 2.0 DNA:DNA) used by default.  \texttt{-E
    2.0} will show all library sequences with scores with an
  expectation value $<=$ 2.0.  With \texttt{fasta-36}, a second
  value, \texttt{e\_cut\_r} is available to limit the E()-values of
  additional sequence alignments between the query and library
  sequences.  If not given, the threshold is \texttt{e\_cut}/10.0.  If
  given with a value $>$ 1.0, \texttt{e\_cut\_r} = \texttt{e\_cut} /
  value; for a value $<$ 1.0, \texttt{e\_cut\_r} = value; If
  \texttt{e\_cut\_r} $<$ 0, then the additional alignment option is
  disabled.
\item[\texttt{-f \#}]
Gap open penalty (-10 by default for proteins,
-12 for DNA, -12 for \texttt{[t]fast[xy]}).
\item[\texttt{-F \#}]
Limit the number of scores and alignments shown based on the expected
number of scores. \texttt{-E \#} sets the highest E()-value shown; \texttt{-F \#} sets
the lowest E()-value displayed. Thus, \mbox{\texttt{-F 0.0001}} will not show any matches or
alignments with E() $<$ 0.0001.  This allows one to skip over close
relationships to search for more distant relationships.
\item[\texttt{-g \#}]
Penalty per residue in a gap (-2 by default for proteins,
-4 for DNA, -2 for \texttt{[t]fast[xy]}).  A single residue gap costs \texttt{f} $+$ \texttt{g}.
\item[\texttt{-h}]
Short help message. Help options with \texttt{':'}, e.g. \texttt{-s:},
require an argument (\texttt{-s BP62}).  Defaults are shown in square
brackets, e.g.: \texttt{-s:\ [BL50]}.
\item[\texttt{-help}]
Long help message
\item[\texttt{-H}]
Show histogram.
\item[\texttt{-i}]
DNA queries - search with reverse complement.  For 
\texttt{tfastx36/y36}, search the reverse complement of the library sequence
only (complement of \texttt{-3} option).
\item[\texttt{-I}]
Interactive mode (the default for versions older than \texttt{fasta-36.3.4}).
\item[\texttt{-j \#}]
Penalty for frameshift between codons (\texttt{[t]fastx36}, \texttt{[t]fasty36}) and within a codon (\texttt{fasty36}/ \texttt{tfasty36} only).
\item[\texttt{-J}]
(\texttt{lalign36} only) show the identity alignment (normally
  suppressed, \texttt{-I} in versions before \texttt{fasta-36.3.4}).
\item[\texttt{-k \#}]
number of shuffles for statistical estimates from shuffling.
\item[\texttt{-l file}]
Location of library menu file (FASTLIBS).
\item[\texttt{-L}]
Display longer library sequence description.
\item[\texttt{-M low-high}]
Range of amino acid sequence lengths to be included in the search.
This option \textit{excludes} library (subject) sequences outside of the specified
range.  In addition, it can be used to \textit{include} library
sequences that would not normally be aligned by \texttt{ggsearch},
because \texttt{ggsearch} assumes that for global alignments, the two
sequences should be approximately the same length (between 0.75 and
1.33 the query sequence length).  Setting \texttt{-M 1-100000} removes
the library sequence length filter for \texttt{ggsearch}.
\item[\texttt{-m \#}]
Specify alignment type: 0, 1, 2, 3, 4, 5, 6, 8, 9, 10, 11, B, BB, ``F\# out\_file''
\begin{small}
\begin{verbatim}
    -m 0        -m 1          -m 2          -m 3        -m 4
MWRTCGPPYT   MWRTCGPPYT    MWRTCGPPYT                 MWRTCGPPYT
::..:: :::     xx  X       ..KS..Y...    MWKSCGYPYT   ----------
MWKSCGYPYT   MWKSCGYPYT
\end{verbatim}
\end{small}
If the \texttt{-V '*@\%'} annotation option has been used, 
annotations can be included in either the coordinate line (the default) or the
middle alignment line (\texttt{-m 0M}, \texttt{-m 1M}), or both
(\texttt{-m 0B}, \texttt{-m 1B}).  See \texttt{-V} for more details.

\indent \texttt{-m 5}: a combination of \texttt{-m 4} and \texttt{-m
  0}. \texttt{-m 6} provides \texttt{-m 5} plus HTML formatting.  In
addition, independent \texttt{-m} options can be combined. Thus, one
can use \texttt{-m 1 -m 6 -m 9}.

\item[\texttt{-m 8}] provides BLAST tabular format output (a tab
  delimited line with the query name, library name, percent identity,
  and other alignment information). ``\texttt{-m 8C}'' provides the
  additional information provided by the BLAST tabular format with
  comment lines.  BLAST tabular format has been extended to include
  either a CIGAR string alignment encoding (\texttt{-m 8CC} with BLAST
  comments, \texttt{-m 8XC} without comments) and, if available, an
  annotation encoding matching FASTA \texttt{-m 9C} output. \texttt{-m
    8CB} provides the BLAST BTOP encoding, rather than the CIGAR
  encoding provided by \texttt{-m 8CC}.  All the \texttt{-m 9c/C/d/D}
  encodings are available with BLAST tabular output using \texttt{-m
    8C[c/C/d/D]}.  In addition to the BLAST BTOP/CIGAR encoding field
  added with \texttt{-m 8CB/8CC}, a second additional annotation field is added if annotations are available. The annotation field provides the same annotation boundary and score information as in standard alignment, but in a more compact form:
\begin{verbatim}
  |RX:3-82:3-82:s=780;b=201.;I=0.866;Q=570.;C=C.Thio|XR:3-82:3-82:s=780;b=201.;I=0.866;Q=570.5;C=C.Thio...
\end{verbatim}
  \texttt{|RX} annotations are based on the query sequence; \texttt{|XR} on the subject. In the example above, the boundaries of the domain in the query and subject are shown (\texttt{3-82:3-82}, as well as the raw score, bit-score, identity, Q-score, and domain name.

  In the v36.3.8h release, 
  new options have been added to \texttt{-m 8CB}: (a) \texttt{-m 8CBl} to show the query and subject sequence lengths;
  (b) \texttt{-m8CBs} to show percent similarity in addition to percent identity; (c) \texttt{-m8CBd} adds an additional output field that shows raw domain information. \texttt{-m8CBL} provides both query/subject sequence lengths and raw domain informatio.

 The \texttt{- m8CBl} option adds the lengths of
  the query and subject sequences after the \texttt{seqid}'s to BLAST
  tabular output, e.g. \texttt{qseqid qlen sseqid slen percid ...}

  \texttt{-m 8CBs}. The \texttt{- m8CBs} includes percent similarity after percent identity,
  e.g. \texttt{qseqid, sseqid \% identity, \% similar, ... }

  \texttt{-m 8CBd} adds a new field with
  the raw domain coordinate information for the query and subject, with \texttt{DX/XD} used to indicate the domains.
  \texttt{-m 8CBL} provides both query/subject lengths and raw domain information.


\begin{verbatim}
  |DX:3-82;C=C.Thio|DX:105-201;C=C.GST_C|XD:3-82;C=C.Thio|XD:105-201;C=C.GST_C
\end{verbatim}

  \texttt(-m 8CBr) (v36.3.8i release, 14-Nov-2022) adds a new field with the raw optimal alignment score (typically Smith-Waterman for proteins).

  These \texttt{-m 8CB} options, \texttt{'l/L'}, \texttt{'s'}, \texttt{'d'}, and \texttt{'r'} can be combined in any order.

\item[\texttt{-m 9}] display alignment coordinates and scores with the
  best score information.  \texttt{-m 9i} provides alignment length,
  percent identity, and percent similarity only. \texttt{-m 9i} also
  provides variation information if it improves the score. \texttt{-m
    9I} provides both identity and domain information on the summary
  line.

 \texttt{ -m 9, -m 9c} and \texttt{-m 9C} extend the normal best score information:
\begin{footnotesize}
\begin{verbatim}
The best scores are:                                      opt bits E(14548)
XURTG4 glutathione transferase (EC 2.5.1.18) 4 -   ( 219) 1248 291.7 1.1e-79
\end{verbatim}
\end{footnotesize}

to include the additional information (on the same line, separated by
$<$tab$>$ characters):
\begin{footnotesize}
\begin{verbatim}
%_id  %_gid   sw  alen  an0  ax0  pn0  px0  an1  ax1 pn1 px1 gapq gapl  fs
0.771 0.771 1248  218    1  218    1  218    1  218    1  219   0   0   0
\end{verbatim}
\end{footnotesize}

The first two values are fraction identical and fraction similar
(score $\ge 0$), followed by the Smith-Waterman alignment score (\texttt{sw}), the
alignment length (\texttt{alen}), and the coordinates of the beginning
and end of the alignment in the query and target (library) sequences
(\texttt{an0} beginning, \texttt{ax1} end in query; \texttt{an1}
beginning, \texttt{ax1} end in target/library), and the coordinate
system for the beginning and end of the query and target/library
sequence (\texttt{pn0} is the displayed coordinate of the first
residue of the query sequence, \texttt{px0} is the displayed
coordinate of the last residue, \texttt{pn1},\texttt{px1} provide the
coordinates for the target/library sequence). \texttt{gapq},
\texttt{gapl} report the number of gaps in the query and library
sequence; \texttt{fs} reports the number of frameshifts.

\texttt{ -m 9c} provides additional information: an encoded alignment string.  For example, the alignment:
\begin{footnotesize}
\begin{verbatim}
       10        20        30        40        50          60         70  
GT8.7  NVRGLTHPIRMLLEYTDSSYDEKRYTMGDAPDFDRSQWLNEKFKL--GLDFPNLPYL-IDGSHKITQ
       :.::  . :: ::  .   .:::         : .:    ::.:   .: : ..:.. :::  :..:
XURTG  NARGRMECIRWLLAAAGVEFDEK---------FIQSPEDLEKLKKDGNLMFDQVPMVEIDG-MKLAQ
               20        30                 40        50        60        
\end{verbatim}
\end{footnotesize}
would be encoded:
\begin{footnotesize}
\texttt{=23+9=13-2=10-1=3+1=5}
\end{footnotesize}.
The numbers in the alignment encoding is with repect to the beginning
of the alignment, not the sequences.  The beginning coordinate of the
alignment is given earlier in the \texttt{-m 9c} line.  \texttt{-m 9C}
provides the alignment encoding in CIGAR format:
\begin{footnotesize}
\texttt{28M9D13M2I10M1I3M1D5M}
\end{footnotesize}.

(June, 2014) The \texttt{-m 9c/C} option has been extended to
\texttt{-m 9d/D}, which encodes the positions of mismatches as well as
insertions and deletions.  For the example above, the \texttt{-m 9d}
encoding would be:
\begin{footnotesize}
\texttt{=1x1=2x4=2x1=2x7=3-9=1x2=1x4=2x1=1x1+2x1=1x1=1x3=1x2+1=3-1x1=1x2=1}
\end{footnotesize}
while \texttt{-m 9D} would be:
\begin{footnotesize}
\texttt{1M1X2M4X2M1X2M7X3M9D1M2X1M4X2M1X1M1X2I1X1M1X1M3X1M2X1I3M1D1X1M2X1M}
\end{footnotesize}
\item[\texttt{-m 10}]
a parseable format for use with other programs (this option no longer reliably tested; \texttt{-m 8CBL} is easier to parse and tested more extensively).
\item[\texttt{-m 11}]
Provide \texttt{lav}-like output (used by \texttt{lalign}) for graphical output.
\begin{quote}
\texttt{lalign36 -m 11 mchu.aa mchu.aa | lav2plt.pl --dev ps  > mchu\_laln.ps}
\end{quote}
Produces a postscript plot of the local alignments.  Likewise,
\texttt{lav2plt.pl --dev svg} produces SVG output.

\item[\texttt{-m BB}] Format output to mimic BLAST format.  \texttt{-m
  B} formats alignments to look like BLAST alignments (Query/Sbjct),
  but is FASTA output otherwise. \texttt{-mBB} imitates BLAST as much
  as possible, and cannot be used with other \texttt{-m} options.

\item[\texttt{-m "F\# out.file"}] Send an alternate result format to \texttt{out.file}.
Normally, the \texttt{-m out\_fmt} option applies to the default output
file, which is either \texttt{stdout}, or specified with \texttt{-O out\_file} (or within
the program in interactive mode). With \texttt{-m F}, an output format can be
associated with a separate output file, which will contain a complete
FASTA program output.  Thus,
\begin{quote}
\begin{small}
\begin{verbatim}
  ssearch36 -m 9c -m "FBB blast.out" -m "F9c,10 m9c_10.out" query library
\end{verbatim}
\end{small}
\end{quote}
Sends the \texttt{-m 9c} output to \texttt{stdout}, but will also send
\texttt{-m BB} output to the \texttt{blast.out} file, and \texttt{-m 9c -m
  10} output to \texttt{m9\_c10.out}.  Consistent \texttt{-m out\_fmt}
commands can be set to the same file by separating them with ','.
Producing alternative format alignments in different files has little
additional computational cost.

Because a space (\textvisiblespace) is used to separate the output
format (\texttt{-m}) values from the file name, the \texttt{-m F}
argument must typically be surrounded by quotation marks (\texttt{"}).

One of the shortcomings of this approach is that it affects only the
output format, not the other options that modify the amount of output.
Thus, if you specify \texttt{-E 0.001}; that expect threshold will be
used for all the output files.  When a \texttt{-m} option does modify
the output (e.g. \texttt{-m 8} sets \texttt{-d 0}), that modification
is specific to the output file.

\item[\texttt{-M low-high}]
Include library sequences with lengths between low and
high.
\item[\texttt{-n}]
Force the query sequence to be treated as a DNA sequence.  
Useful when query sequences contain a large number of
ambiguous residues, e.g. transcription factor binding sites.
\item[\texttt{-N \#}]
break long library sequences into blocks of \# residues.  Useful for
bacterial genomes, which have only one sequence entry.  -N 2000 works
well for well for bacterial genomes. (This option was required when
FASTA only provided one alignment between the query and library
sequence.  It is not as useful, now that multiple alignments are
available.)

\item[\texttt{-o off1,off2}]
(Previously \texttt{-X}.) Specifies offsets for the beginning of the query and library sequence.
For example, if you are comparing upstream regions for two genes, and
the first sequence contains 500 nt of upstream sequence while the
second contains 300 nt of upstream sequence, you might try:
\begin{quote}
\texttt{fasta -o "-500 -300" seq1.nt seq2.nt}
\end{quote}
If the \texttt{-o} option is not used, FASTA assumes numbering starts with 1.
(You should double check to be certain the negative numbering works
properly.)

\item[\texttt{-O}] Send a copy of results to \texttt{filename}.
  Helpful for environments without STDOUT, but should be avoided (use
  \texttt{> filename} instead).

\item[\texttt{-p}]
Force query to be treated as protein sequence.

\item[\texttt{-P PSSM\_file}]
Specify a PSI-BLAST format PSSM (Position Specific Scoring Matrix)
file.  \texttt{ssearch36}, \texttt{ggsearch36}, and
\texttt{glsearch36} can use a PSSM file to improve the sensitivity of
a search. The FASTA programs accept two PSSM file formats:\\[2ex]
\begin{tabular}{l l l}
\hline\\[-1.5ex]
format & \texttt{blastpgp} & option \\[0.5ex]
\hline\\[-1.5ex]
0 & \texttt{blastpgp -C pssm.chk -u 0} & byte-encoded \\
2 & \texttt{blastpgp -C pssm.asnb -u 2} & binary ASN.1 \\
%  & \texttt{psiblast -out\_pssm\_text} \\[1.5ex]
\hline\\[-0.5ex]
\end{tabular}\\
which can be specified after the file name, e.g.:
\begin{quote}
\texttt{ssearch36 -P 'pssm.asnb 2' pssm\_query.aa +sp+}
\end{quote}
Searches with a PSI-BLAST PSSM must still require a query sequence
file, and the query sequence file must match the PSSM seed sequence.
The format 0 byte-encoded PSSM is machine dependent; it must be
created by \texttt{blastpgp} on the same architecture as
\texttt{ssearch36}.  In general, you should use the binary ASN.1 (format 2) file.

With the release of \texttt{NCBI-BLAST+}, \texttt{psiblast} replaces
\texttt{blastpgp}, and \texttt{psiblast} does not produce the binary
ASN.1 PSSM checkpoint data.  However, the text ASN.1 PSSM checkpoint
file (produced with the \texttt{psiblast} option \texttt{-out\_pssm})
can be converted to a binary ASN.1 format that \texttt{ssearch36} can
read using the NCBI \texttt{datatool} program (available from
\url{ftp://ftp.ncbi.nlm.nih.gov/toolbox/ncbi_tools++/BIN/CURRENT/datatool})
together with
\url{http://www.ncbi.nlm.nih.gov/data_specs/asn/NCBI_all.asn}. More
information about \texttt{datatool} is available from
\url{http://www.ncbi.nlm.nih.gov/data_specs/NCBI_data_conversion.html}.
The NCBI BLASTP/PSI-BLAST website provides the same PSSM text ASN.1
file with the downloads link.  A text ASN.1 PSSM file can be converted
to a binary ASN.1 file using the command:
\begin{quote}
\texttt{datatool -m NCBI\_all.asn -v pssm.asn\_txt -e pssm.asnb}
\end{quote}
The \texttt{pssm.asnb} can then be used with
\texttt{ssearch36} with the \texttt{-P 'pssm.asnb 2'}
option shown above.

\item[\texttt{-Q,-q}]
Quiet - does not prompt for any input.  Writes scores and alignments
to the terminal or standard output file (on by default, turned off
with \texttt{-I}).
\item[\texttt{-r +n/-m}]
Specify match/mismatch scores for DNA comparisons.  The default is
\texttt{+5/-4}. \texttt{+3/-2} can perform better in some cases.
\item[\texttt{-R file}]
Save a results summary line for every sequence in the sequence
library.  The summary line includes the sequence identifier,
superfamily number (if available) position 
in the library, and the similarity scores calculated.  This option can
be used to evaluate the sensitivity and selectivity of different
search strategies \cite{wrp951,wrp981}.
\item[\texttt{-s file}] Specify the scoring matrix file.
  \texttt{fasta36} uses the same scoring matrice format as Blast.
  Several scoring matrix files are included in the standard
  distribution in the \texttt{data/} directory.  For protein
  sequences: \texttt{codaa.mat} - based on minimum mutation matrix;
  \texttt{idnaa.mat} - identity matrix; \texttt{pam250.mat} - the
  PAM250 matrix; \cite{day787}, (\texttt{-s P250}), and
  \texttt{pam120.mat} - a PAM120 matrix (\texttt{-s P120}).  The
  default scoring matrix is BLOSUM50 (\texttt{-s BL50}). Other
  matrices include a series of modern PAM-based matrices
  \cite{tay925}: MDM40/\texttt{-s MD40}, MDM20/\texttt{-s MD20}, and
  MDM10/\texttt{-s MD10}, and a selection from the BLOSUM series
  \cite{hen929} BLOSUM50, 62, and 80/\texttt{-s BL50}, \texttt{-s
    BL62}, \texttt{-s BL80}.  \texttt{-s BP62} sets the scoring matrix
  to BLOSUM62 and the gap penalties to -11/-1, identical to
  \texttt{BLASTP}.  In addition, the VTML160 matrix (\texttt{-s
    VT160}) \cite{muller2002} and OPTIMA\_5 (\texttt{-s OPT5})
  \cite{kan023} are available.

If the scoring matrix is prefaced by a question mark,
e.g. \texttt{?BP62}, then the scoring matrix is adjusted for each
query to ensure that a 100\% identical match can produce a score of at
least 40 bits.  This is designed for \texttt{fastx36} searches with
potentially short DNA queries; A 120 nt DNA query can only produce a
40 amino-acid alignment, which, with BLOSUM62 -11/-1, cannot produce
more than 23 bits of score. A scoring matrix with a higher information
content is required; in the set available by default, MD40, with 2.22
bits/position, would be used.  For more information about alignment
length and information content, see \cite{alt915}.

\item[\texttt{-S}] Filter out lower-case characters in the query or
  library sequences for the initial score calculation (used to filter
  low-complexity -- \texttt{seg}-ed -- residues).  The \texttt{pseg}
  program \cite{woo935} can be used to lower-case mask low complexity
  regions in protein sequences. With the \texttt{-S} option, lower
  case characters in the query or database sequences are treated as
  \texttt{X}'s during the initial scan, but are treated as normal
  residues during the final alignment display.  Since statistical
  significance is calculated from the similarity score calculated
  during the library search, the lower case residues do not contribute
  to the score.  However, if a significant alignment contains low
  complexity regions, the residues are shown (as lower
  case characters, Fig. \ref{seg-aln}).

The \texttt{pseg} program can be used to produce databases (or query
sequences) with lower case residues indicating low complexity regions
using the command:
\begin{verbatim}
pseg ./swissprot.fasta -z 1 -q > swissprot.lseg
\end{verbatim}

The \texttt{-S} option should always be used with \texttt{FASTX/Y} and
\texttt{TFASTX/Y} because out-of-frame translations often generate
low-complexity protein sequences.  However, only lower case characters
in the protein sequence (or protein database) are masked; lower case
DNA sequences are translated into upper case protein sequences, and
not treated as low complexity by the translated alignment
programs. (There is an option in the \texttt{Makefile},
\texttt{-DDNALIB\_LC}, to enable preserving case in DNA sequences.)

\item[\texttt{-t \#}] Translation table - fastx36, tfastx36, fasty36,
  and tfasty3 now support the BLAST translation tables.  See
  \url{http://www.ncbi.nlm.nih.gov/Taxonomy/Utils/wprintgc.cgi}.  

  \texttt{-t 1} also enables translation of \texttt{'TGA'} to
  \texttt{'U'} (seleno-cysteine) (by default, \texttt{'TGA'} is
  translated to \texttt{'*'}). Because of the ambiguity of the
  \texttt{'TGA'} codon, translated alignments of \texttt{'TGA'} with
  \texttt{-t 1} match \texttt{'U'} and \texttt{'*'} (termination)
  equally well.

\texttt{-t t} enables the addition of an implicit termination codon to
a protein:translated DNA match.  That is, each protein sequence
implicitly ends with \texttt{*}, which matches the termination codes
for the appropriate genetic code.  To change the translation table and
insert a termination character after each protein sequence, use
\texttt{-t 1 -t t}.

\item[\texttt{-T \#}]
set number of threads/workers.  Normally on a multi-core machine, the maximum
number of processors/cores is used.
\item[\texttt{-U}]
Treat the query sequence an RNA sequence.  In addition to selecting a
DNA/RNA alphabet, this option causes changes to the scoring matrix so
that \texttt{G:A} , \texttt{T:C} or \texttt{U:C} are scored as \mbox{\texttt{G:G -3}}.
\item[\texttt{-v \#}]
Do window shuffles with the window size specified.
\item[\texttt{-V str}] Specify annotation characters that can be
  included (and will be ignored), in the query sequence file, but are
  displayed in the alignments.  If a query file contains
  \texttt{"ACVS*ITRLFT?"}, where \texttt{"*"} and \texttt{"?"}  are
  used to indicate phosphorylation, giving the option \mbox{\texttt{-V
      '*?'}}, the annotated characters in the query will (\texttt{S*},
  \texttt{F?}) will be highlighted in the alignment (on the number
  line). A \texttt{fasts36} alignment of \texttt{seq/ngts.aa} compared
  to \texttt{seq/mgstm1.aa} with \texttt{-V '*?'} produces:
\begin{footnotesize}
\begin{verbatim}
             *                10??                                 
GT8.7     ILGYWN------------EYTDSSYDEKR----------------------------
          ::::::            :::::::::::                            
GT8.7  MPMILGYWNVRGLTHPIRMLLEYTDSSYDEKRYTMGDAPDFDRSQWLNEKFKLGLDFPNL
               10        20        30        40        50        60
\end{verbatim}
\end{footnotesize}
In addition to showing the alignments of post-translationally modified
sites, the \texttt{-V} option can be used to highlight active sites in
library sequences. In the \texttt{-m 9c} output, the state of the
annotated sites is summarized when \texttt{-V} is used.

(fasta-36.3.6 June 2012) The \texttt{-V} option has been extended to:
(1) allow feature descriptions to be specified in a file,
e.g. \texttt{-V =annot.defs} where \texttt{annot.defs} contains:
\begin{footnotesize}
\begin{verbatim}
*:phosphorylation
@:active site
^:binding site
\end{verbatim}
\end{footnotesize}
The annotation character is left of the  ':', the definition is on the
right.  The \texttt{annot.defs} file can also be specified by setting
the \texttt{FA\_ANNOT\_DEF} environment variable to the file name;

(2) to include optional annotation file, e.g. \texttt{-V
  '<features.annot'}, or script, e.g. \texttt{-V '!features.pl'} for
library annotations and \texttt{-V 'q!features.pl'} for query annotations. (Some shells require \texttt{\textbackslash!features.pl}.) Similar to the library expansion script, the
\texttt{features.pl} script is run against a temporary file containing
the list of high scoring sequence accessions (the text before the
first space), e.g.
\begin{footnotesize}
\begin{verbatim}
sp|P09488.3|GSTM1_HUMAN
sp|Q03013.3|GSTM4_HUMAN
sp|Q5R8E8.3|GSTM2_PONAB
...
\end{verbatim}
\end{footnotesize}
The \texttt{features.pl} script then produces a file of annotations on
those sequences, in the format:
\begin{verbatim}
>accession1
position label value
>accession2
...
\end{verbatim}
For example:
\begin{footnotesize}
\begin{verbatim}
>sp|P09488.3|GSTM1_HUMAN
23	*	
33	*	
34	*	
116	^	
173	V	N
210	V	T
>sp|Q03013.3|GSTM4_HUMAN
2	V	P
116	^	
160	V	V
208	V	V
209	V	F
211	V	K
212	V	M
>sp|Q5R8E8.3|GSTM2_PONAB
...
\end{verbatim}
\end{footnotesize}
The same format is used for the \texttt{-V '<feature.annot'} file.

The \texttt{V} label is special; it indicates that the feature is a
variant residue and specifies the alternative residue in the label
field.  Thus, \texttt{GSTM4\_HUMAN} can have a \texttt{M} at position
2.  Unlike modification or active site annotations, variant residues
can change the sequence of the library sequence if replacing the
canonical library residue with the variant residue improves the score.
Thus, without the \texttt{-V '!feature.pl'} script, the human
\texttt{GSTM1B} variant with dbSNP:rs449856 would align to
\texttt{GSTM1\_HUMAN} (\texttt{P04988}) like this (the \texttt{-m 1}
format option was used to highlight differences) :
\begin{footnotesize}
\begin{verbatim}
The best scores are:                                                          opt bits E(1)
sp|P09488.3|GSTM1_HUMAN Glutathione S-transferase Mu 1; GST HB subuni  ( 218) 1490 335.9 3.7e-97

>>sp|P09488.3|GSTM1_HUMAN Glutathione S-transferase Mu 1; GST HB s            (218 aa)
 initn: 1490 init1: 1490 opt: 1490  Z-score: 1776.8  bits: 335.9 E(1): 3.7e-97
Smith-Waterman score: 1490; 99.1% identity (100.0% similar) in 218 aa overlap (1-218:1-218)

...
              170       180       190       200       210        
gtm1_h YDVLDLHRIFEPNCLDAFPNLKDFISRFEGLEKISAYMKSSRFLPRPVFTKMAVWGNK
                   x                                    x        
sp|P09 YDVLDLHRIFEPKCLDAFPNLKDFISRFEGLEKISAYMKSSRFLPRPVFSKMAVWGNK
              170       180       190       200       210        
\end{verbatim}
\end{footnotesize}

With a \texttt{-V '!feature.pl'} script to annotate the variants,
the alignment becomes:
\begin{footnotesize}
\begin{verbatim}
The best scores are:                                                          opt bits E(1)
sp|P09488.3|GSTM1_HUMAN Glutathione S-transferase Mu 1; GST HB s       ( 218) 1500 338.1   8e-98

>>sp|P09488.3|GSTM1_HUMAN Glutathione S-transferase Mu 1; GST HB s            (218 aa)
 Variant: K173N;S210T;
 initn: 1500 init1: 1500 opt: 1500  Z-score: 1788.7  bits: 338.1 E(1): 8e-98
Smith-Waterman score: 1500; 100.0% identity (100.0% similar) in 218 aa overlap (1-218:1-218)
...
              170       180       190       200       210        
gtm1_h YDVLDLHRIFEPNCLDAFPNLKDFISRFEGLEKISAYMKSSRFLPRPVFTKMAVWGNK
                                                                 
sp|P09 YDVLDLHRIFEPNCLDAFPNLKDFISRFEGLEKISAYMKSSRFLPRPVFTKMAVWGNK
              170  V    180       190       200       21V        
\end{verbatim}
\end{footnotesize}
In addition to removing the two differences as residues 173 and 210,
which produces a 100\% identical alignment, inclusion of the variant
library sequence also improves the raw similarity score and $E()$-value.
An example script (\texttt{misc/up\_feats.pl}) that extracts
annotations from a mysql database of Uniprot features is provided.

If the annotation script produces lines beginning with '=', then these
lines are taken as annotation definitions, similar to the
\texttt{annot.defs} file described above.  Thus:
\begin{footnotesize}
\begin{verbatim}
=*:phosphorylation
=@:active site
=^:binding site
>sp|P09488.3|GSTM1_HUMAN
23	*	
33	*	
34	*	
116	^	
173	V	N
210	V	T
\end{verbatim}
\end{footnotesize}
will produce the same annotation descriptions as the
\texttt{annot.defs} file.

Scripts to produce annotations are available in the \texttt{scripts/}
directory as \texttt{scripts/ann\_feats*.pl}. Scripts with
\texttt{www} in the name,
e.g. \texttt{scripts/ann\_feats\_up\_www2.pl} and
\texttt{scripts/ann\_pfam\_www.pl} download annotation information
from Uniprot or Pfam web services, respectively.  Scripts lacking
\texttt{www} require require a MySQL database that associates features
or domains with sequence identifiers (accessions).  With \CURRENT,
domain annotations are allows to overlap each other (which often
happens in Pfam and UniProt); FASTA 36.3.6 did not support overlapping
domains.  Scripts that can produce overlapping domain annotations have
\texttt{\_e} in their names, but will produce non-overlapping domain
annotations with the \texttt{--no-over} option. Thus:
\texttt{scripts/ann\_pfam\_www\_e.pl --acc sp|P43553|ALR2\_YEAST}
produces:
\begin{quote}
\begin{verbatim}
>sp|P43553|ALR2_YEAST
451  -  683   PF01544 :1
667  -  799   PF01544 :1
\end{verbatim}
\end{quote}
While \texttt{scripts/ann\_pfam\_www\_e.pl --acc --no-over
  sp|P43553|ALR2\_YEAST} produces:
\begin{quote}
\begin{verbatim}
>sp|P43553|ALR2_YEAST
451  -  675  PF01544 :1
676  -  799  PF01544 :1
\end{verbatim}
\end{quote}

\item[\texttt{-w \#}] 
  Display width value ($<$200). Sets the approximate width of the
  high-score descriptions and the length of residue
  alignments. \texttt{-w 60} by default.

\item[\texttt{-W \#}] context length (default is 1/2 of line width -w)
  for alignment, for programs like \texttt{fasta36} and
  \texttt{ssearch36}, that provide additional sequence context.

\item[\texttt{-X extended\_option}]
A number of rarely used options are now only available as extended options:

\begin{description}

\item[\texttt{X1}] sort output by \texttt{init1} score (for
  compatibility with FASTP; obsolete).

\item[\texttt{XA}] force banded Smith-Waterman final alignment for proteins

\item[\texttt{Xa}] only report annotation information in -m 8CB output.

\item[\texttt{XB}] Calculate pecent identity, percent similarity, and
  alignment using the BLAST model, which excludes gapped residues.
  This allows very high identity alignments with large gaps to look
  much closer, but causes the alignment length to drop by the length
  of the gap.

\item[\texttt{Xb}] (Previously \texttt{-B}.)  Show the z-score, rather
  than the bit-score in the list of best scores (rarely used, provided
  for backward compatibility).

\item[\texttt{Xg}] (gi\_save) Historically, FASTA and BLAST have
  treated the string \texttt{>gi|123456} specially, removing it from the list
  of high scoring sequences and alignment output.  \texttt{-Xg}
  causes \texttt{gi} numbers to be displayed.

\item[\texttt{XI}] Modify rounding used in percent identity/percent
  similarity display to ensure that sequences that have a mismatch are
  not shown as 100.0\% identical.  Without this option, a single
  mismatch in a 10,000 residue alignment would be shown as 100.0\%
  identical; with this option, it would be shown as 99.9\%
  identical.

\item[\texttt{Xo}] (\texttt{fasta36}, \texttt{[t]fast[x/y]36} only)
  (Previously \texttt{-o}.) Turn off the default \texttt{opt} score
  calculation and sort results by \texttt{initn} scores (reduces
  sensitivity and statistical accuracy, obsolete).

\item[\texttt{XM}] The maximum amount of memory available for storing
  the library in multi-sequence searches. The value is specified in
  MBytes (\texttt{-XM16}) or GBytes (\texttt{-XM4G}) and can also be
  set using the \texttt{LIB\_MEMK} environment variable
  (\texttt{LIB\_MEMK=4G}).  Negative values remove the memory
  restriction. By default (set as a compile-time option,
  \texttt{-DMAX\_MEMK=2}), set to 2 GBytes in 32-bit environments and
  12 GBytes in 64-bit environments.

\item[\texttt{XN/XX}] Alter the treatment of N:N (DNA) or X:X
  (protein) alignments for counts of identities and similarities. By
  default the \texttt{FASTA} programs count N:N or X:X as identical,
  but not similar, because their alignment scores are typically
  negative. \texttt{-XNS}, \texttt{-XN+}, \texttt{-XXS}, and
  \texttt{-XX+} treat N:N and X:X alignments as ``similar'' , even
  though their alignment scores are negative, when calculating percent
  similarity. \texttt{-XND}, \texttt{-XN-}, \texttt{-XXD}, and
  \texttt{-XX-} treat N:N and X:X alignments as non-identical for
  calculating percent identity.

\item[\texttt{Xx}] (Previously \texttt{-x}) Specify the penalty for a
  match to an \texttt{X}, and mismatch to \texttt{x}, independently of
  the PAM matrix.  Particularly useful for \texttt{fastx3/fasty36},
  where termination codons are encoded as \texttt{X}.  For example,
  \texttt{-Xx=0,-1} scores an \texttt{X:X} match as 0, and
  \texttt{X:not-X} as -1.

\item[\texttt{Xy}] (Previously \texttt{-y}.) Set the width of the band
  used for calculating "optimized" scores.  For proteins and ktup=2,
  the width is 16.  For proteins with ktup=1, the width is 32 by
  default.  For DNA the width is 16.

\end{description}

\item[\texttt{-z -1,0,1,2,3,4,5,6}]\hfill\\
\texttt{-z -1} turns off statistical calculations. \texttt{z 0} estimates
the significance of the match from the mean and standard deviation of
the library scores, without correcting for library sequence length.
\texttt{-z 1} (the default) uses a weighted regression of average score
vs library sequence length; \texttt{-z 2} uses maximum likelihood
estimates of $\lambda$
and $K$; \texttt{-z 3} uses Altschul-Gish parameters \cite{alt960};
\texttt{-z 4 - 5} uses two variations on the \texttt{-z 1}
strategy. \texttt{-z 1} and \texttt{-z 2} are the best methods, in
general.
\item[\texttt{-z 11,12,14,15,16}]\hfill\\
estimate the statistical parameters from shuffled copies of each
library sequence.  This allows accurate statistics to be estimated for libraries comprised of a single protein family.

\item[\texttt{-z 21,22,24,25,26}]\hfill\\
estimate the statistical parameters from shuffled copies of the
highest scoring sequences reported in the search.
library sequence. This shuffling strategy is much more like
\texttt{prss}, since the sequences shuffled share compositional
similarity to the query.
\item[\texttt{-Z db\_size}]
sets the apparent size of the database to be used when calculating
expectation E()-values.  If you searched a database with 1,000
sequences, but would like to have the E()-values calculated in the
context of a 100,000 sequence database, use \texttt{-Z 100000}.
\item[\texttt{-3}]
translate only three forward frames or search with only the forward
strand (complement of \texttt{-i}).
\end{description}

Thus, to tell \texttt{fasta36} to align \texttt{seq1.aa} with \texttt{seq2.aa} showing the entirety of both sequences, with 80 characters per line, one would type:
\begin{verbatim}
fasta36 -w 80 -s BP62 -a seq1.aa seq2.aa
\end{verbatim}
The \texttt{-w 80} and \texttt{-a} options must precede the file
names.  If you just enter the options on the command line followed by
\texttt{-I}, the program will prompt for the file names.

In addition, the FASTA programs can accept query sequence data from
\texttt{STDIN}.  To specify that stdin be used as the query or library
file, the file name should be specified as \texttt{@}.  Thus:
\begin{quote}
\texttt{cat query.aa | fasta36 @:25-75 /slib/swissprot }
\end{quote}
would take residues 25-75 from \texttt{query.aa} and search the
\texttt{/slib/swissprot}.

\subsubsection{Environment variables}

FASTA allows virtually every option to be set on the command line
(except the \emph{ktup}, which must be set as the third command line
argument), but it is often convenient to set the \texttt{FASTLIBS}
environment variable to specify the location of the \texttt{fastlibs}
database description file.

\texttt{FASTLIBS} -- \texttt{FASTLIBS}
specifies the location of the file that contains the list of library
descriptions, locations, and library types (see section on finding
library files).

\texttt{LIB\_MEMK} -- Set the maximum amount of memory (MBytes) to be
available for library buffering (equivalent to \texttt{-XM\#}, see
above).  By default, \texttt{2GB} is available on 32-bit systems
(\texttt{LIB\_MEMK=2G}); \texttt{8GB} on 64-bit systems.

\texttt{REF\_URL}, \texttt{SRCH\_URL} and \texttt{SRCH\_URL1} -- These
environment variables are used in HTML mode (\texttt{-m 6}) to provide
links from the sequence alignment (see the links at
\url{http://fasta.bioch.virginia.edu/fasta_www2/}). \texttt{REF\_URL}
is associated with the \texttt{Entrez Lookup} link; \texttt{SRCH\_URL}
with the \texttt{General re-search} link, and \texttt{SRCH\_URL1}
with the \texttt{Pairwise alignment} link.  In each case, the text
corresponds to a HTML URL, but with positions containing the
\texttt{\%s} or \texttt{\%ld} (for numbers) part of a 'C'
\texttt{sprintf()} call for specific variables. \texttt{REF\_URL} uses
the database (\texttt{protein} or \texttt{nucleotide}), together with
a query term (typically the \texttt{gi} number). \texttt{SRCH\_URL}
and \texttt{SRCH\_URL1} use \texttt{db}, \texttt{query} (\texttt{gi},
\texttt{pgm} (\texttt{fa}, \texttt{ss}, \texttt{fx}, etc.), and
\texttt{start}, \texttt{stop}, and \texttt{n1} (library sequence
length), where \texttt{start} and \texttt{stop} are the boundaries of
the alignment, for sub-sequence searches.  The values of these
environment variables are used with \texttt{sprintf} to build a new
URL that is linked in the output.

\texttt{TMP\_DIR} -- Location (if defined) of the temporary files used
by the \texttt{-e expand\_script.sh} option.

In addition, environment variables can be used inside both the
\texttt{fastlibs} file and in the \texttt{@db.nam} files of file
names. The \texttt{fasta36/conf/fast\_libs\_e.www} file, included with
the distribution, shows an example, as do the descriptions of file of
file names files shown below. Whenever a word of the form
\texttt{\$\{WORD\}} is found in \texttt{fastlibs} or a file of file
names, the \texttt{\$\{WORD\}} environment variable is expanded and
inserted in the string.  Thus, if \texttt{<\$\{SLIB\}/blast\_dbs/}
describes where a list of files will be found and \texttt{\$\{SLIB\}}
is \texttt{"/seqdata"}, then the resulting substitution yields:
\texttt{</seqdata/blast\_dbs/}.

\section{Installing FASTA and the sequence databases}

\subsection{Obtaining/preparing the sequence libraries}

The FASTA program package does not include any protein or DNA sequence
libraries.  Protein and DNA sequence databases are available via
anonymous FTP from the NCBI (\url{ftp://ftp.ncbi.nlm.nih.gov/blast/db},
\url{ftp://ftp.ncbi.nlm.nih.gov/blast/db}), UniProt
(\url{ftp://ftp.uniprot.org/pub/databases/uniprot}), and the EBI
(\texttt{ftp.ebi.ac.uk/pub/databases}).

\emph{Protein Sequence Databases} -- Protein sequence databases are
available from the NCBI, UniProt, and the EBI.  The NCBI provides a
``raw'' database, \texttt{nr}, and a well-curated, less redundant
database, \texttt{refseq\_protein}, and a copy of the very well
annotated \texttt{swissprot} database. Protein sequence databases can
also be downloaded from UniProt and the EBI; both sites provide the
same UniProt\cite{uniprot11} database.

Protein libraries, particularly those used for translated-DNA:protein
comparisons with \texttt{fastx36} or \texttt{fasty36}, show be scanned
to remove low-complexity regions.  Matches between low complexity
regions can violate the composition assumptions used by the FASTA
statistical estimates. The \texttt{pseg} program (\cite{woo935},
\url{ftp://ftp.ncbi.nlm.nih.gov/pub/seg/pseg}) can be used to lower-case
low complexity regions, which then can be ignored during the initial
database search by using the \texttt{-S} option.  To lower-case low
complexity regions, run the \texttt{pseg} program against the protein sequence database:
\begin{quote}
\begin{verbatim}
pseg /seqdata/swissprot.fa -z 1 -q > /seqdata/swissprot.lseg
\end{verbatim}
\end{quote}
And then you can run most FASTA programs with \texttt{-S}:
\begin{quote}
\begin{verbatim}
ssearch36 -S mgstm1.aa /seqdata/swissprot.lseg
\end{verbatim}
\end{quote}

Fig. \ref{seg-aln} shows the effect of including the \texttt{-S}
option with lower-cased low-complexity sequences.  The \texttt{opt}
score (407), which is used to sort the results and calculate
statistics, is lower than the Smith-Waterman score (451), even though
exactly the same residues are aligned for each score. The \texttt{opt}
score excludes residues 19-30, because they were marked as
low-complexity by \texttt{pseg}; thus they are shown as lower-case.
The Smith-Waterman score includes the contribution from that part of
the alignment.

Out-of-frame translated DNA sequences often produce low-complexity
regions \cite{wrp973}, so it is particularly important to avoid
low-complexity alignments when using \texttt{fastx36} and
\texttt{fasty36}

\subsection{Searching taxonomic subsets}

Because increasing database size reduces search sensitivity (an
alignment with an $E()$-value of $0.001$ in a search of a 100,000
entry database will have an $E()$-value of 0.1, not significant, if
found in a database of 10,000,000 sequences), it is much more
effective to search smaller, less redundant databases (you can always
search the larger database later).  Thus, the \texttt{refseq\_protein}
database from the NCBI is preferred over \texttt{nr}; even better are
databases that reflect a limited phylogenetic range
(e.g. \texttt{refseq\_human} for vertebrate sequences).

While the NCBI provides organism-specific \texttt{refseq} subsets on
their FTP site, they can be difficult to find.  Alternatively, you can
use the NCBI \texttt{Entrez} web site to download a list of
\texttt{gi} numbers specific to a particular organism or taxonomic
range. The FASTA programs can search a subset of a large sequence
database that is specified by a list of \texttt{gi} numbers by using
library format 10.  For example, given a list of \texttt{gi} numbers
for the human proteins in \texttt{swissprot.lseg}, the file
\texttt{sp\_human.db}, with the content:
\begin{quote}
\begin{verbatim}
<${SLIB}/swissprot.lseg 0:2 4|
3121763
51701705
7404340
205831112
74735515
...
\end{verbatim}
\end{quote}
could be used to search the human subset of
\texttt{swissprot.lseg}. The \texttt{gi} numbers for the SwissProt
entries begin with the second line. The first line specifies the
location of the file where the sequences containing the \texttt{gi}
numbers can be found (\texttt{\$\{SLIB\}/swissprot.lseg}, the
\texttt{libtype} of that file (\texttt{0:fasta}), the character offset
to the beginning of the sequence identifier in that file (\texttt{2}),
the identifier type (\texttt{4}), and the character
that separates the fields in the FASTA descriptor (\texttt{|}).  The
identifier type can take four formats:

\begin{tabular}{l l}
\hline\\[-1.5ex]
1 & ordered accession strings  (letters or numbers)\\
2 & ordered numbers (digits only) \\
3 & un-ordered accession strings \\
4 & un-ordered numbers \\
\hline\\
\end{tabular}\\
(Ordered accession strings/numbers are ordered in both the library and the subset file.)

Thus, given the \texttt{0:2 4|} specification above, the line:
\begin{quote}
\texttt{>sp|O15143.3|ARC1B\_HUMAN Actin-related protein 2/3 ...}
\end{quote}
would be parsed, looking for an number starting at column 4 (the first
column is numbered 0), and ending with \texttt{|}. The order of
sequences in the library do not have to correspond to the order in the
\texttt{sp\_human.db} file (un-ordered). Given a the
\texttt{sp\_human.db} file, a file \texttt{swissprot.lseg} in the
directory specified by the environment variable \texttt{\$\{SLIB\}},
and a command of the form:
\begin{quote}
\texttt{fasta36 -S mgstm1.aa 'sp\_human.db 10'}
\end{quote}
Would use the \texttt{sp\_human.db} file to search the subset of
\texttt{swissprot.lseg} that contained the specified \texttt{gi}
numbers.

\subsection{DNA sequence libraries}

Because of the large size of DNA databases, you will probably want to
keep DNA databases in only one format.  The FASTA3 programs that
search DNA databases --- \texttt{fasta36}, \texttt{fastm36}, and
\texttt{tfastx/y36} --- can read DNA databases in Genbank flatfile (not
ASN.1), FASTA, and BLAST2.0 (\texttt{formatdb}) formats, as well as
EMBL format.  BLAST2.0 format is preferred for DNA sequence libraries,
because the files are considerably more compact than GenBank format.
The NCBI does not provide software for converting from Genbank flat
files to Blast2.0 DNA databases, but you can use the Blast
\texttt{formatdb} program to convert ASN.1 formatted Genbank files,
which are available from the NCBI \texttt{ftp} site.

The NCBI also provides the comprehensive \texttt{nt} DNA database, and
several EST databases in Blast2.0/\texttt{formatdb} format from
\texttt{ftp://ncbi.nlm.nih.gov/blast/db}.


\subsection{Finding the library files}

All the FASTA programs comparison programs have the command line syntax:
\begin{quote}
\texttt{fasta36 query.file /seqdata/library}
\end{quote}
However, in addition to simply specifying the location of the database
to be searched
(\texttt{/seqdata/library}), the FASTA programs
provide several methods for referring to sequence databases without specifying a specific file.  These methods can be used to provide abbreviations for sequence libraries, e.g.:
\begin{quote}
\texttt{fasta36 query.file s} 
or
\texttt{fasta36 query.file +sp+}
\end{quote}
To use abbreviations like \texttt{'s'} or \texttt{'+sp+'} to reference a
sequence database, a \texttt{FASTLIBS} file must be used, see section
\ref{fastlibs}.

Large DNA and protein databases are often distributed across several
files.  For example, the NCBI \texttt{nr} protein database is found in
5 files, \texttt{nr.00} ... \texttt{nr.04}.  To search databases in
multiple files, the names of the files are specified in a file of
filenames, \texttt{nr.nam}:
\begin{quote}
\begin{verbatim}
<${SLIB}/blast_dbs/
nr.00 12
nr.01 12
nr.02 12
nr.03 12
nr.04 12
\end{verbatim}
\end{quote}
In this file, the first line \texttt{<\$\{SLIB2\}/blast\_dbs/},
beginning with \texttt{<}, specifies the location and format (Blast2.0
\texttt{formatdb}) the data files. Text of the form
\texttt{\$\{SLIB\}} refers to Unix/MacOSX/Windows environment
variables; the value of \texttt{\$\{SLIB\}} is set by a Unix/MacOSX
shell environment command. Thus, if the value of \texttt{\$\{SLIB\}}
is \texttt{/seqdata}, then the first sequence library file to be read
will be \texttt{/seqdata/blast\_dbs/nr.00}, in format 12 (Blast2.0
\texttt{formatdb}).

To refer to the \texttt{nr.nam} file as a file of file names, it must
be prefixed by a \texttt{@} character, e.g.
\begin{quote}
\texttt{fasta36 query.file \textbf{@}nr.nam}
\end{quote}
Files of file names can contain references to other files of file names:
\begin{quote}
\begin{verbatim}
<${SLIB}/fasta_dbs/
@pdb.nam
@swissprot.nam
\end{verbatim}
\end{quote}
The FASTA file of file names is similar to the NCBI
\texttt{prot\_db.pal} and \texttt{dna\_db.nal}, files, but
unfortunately they are different, and currently FASTA cannot read NCBI
\texttt{.pal} or \texttt{.nal} files that contain a \texttt{DBLIST}
line. FASTA can read NCBI \texttt{.pal} or \texttt{.nal} files that do
not contain a \texttt{DBLIST} line.

FASTA version \texttt{fasta-36.3.6} provides an alternative way to
generate a database to be searched: the \texttt{!script.sh} file.
Like the \texttt{-e expand\_file.sh} script, a shell script or program
can be used to produce a database to a temporary file, which is then
seached.  For example, if the file \texttt{cat\_db.sh} contains the
command \texttt{echo /seqdb/swissprot.lseg}, the command:
\begin{quote}
\begin{verbatim}
fasta36 query.aa \!@cat_db.sh
\end{verbatim}
\end{quote}
will cause \texttt{cat\_db.sh} to produce a temporary file with the
line \texttt{swissprot.lseg}, which is interpreted as an indirect file
of filenames; thus, because of the \texttt{@}, the file will be
interpreted as an indirect file, and the \texttt{swissprot.lseg} file
will be searched.  Note that on Unix systems, the \texttt{'!'} must be
preceeded by a \texttt{'\textbackslash'} so that it is not interpreted by the
shell, as shown above.

\subsection{\texttt{FASTLIBS}}
\label{fastlibs}

All the search programs in the FASTA3 package can use the environment
variable \texttt{FASTLIBS} to find the protein and DNA sequence
libraries. (Alternatively, you can specify the \texttt{FASTLIBS} file
with the \texttt{-l fastlibs.file} option.) The \texttt{FASTLIBS}
variable contains the name of a file that has the actual filenames of
the libraries.  The \texttt{fastlibs} file included with the
distribution is an example of a file that can be referred to by
FASTLIBS. To use the \texttt{fastlibs} file, type:
\begin{quote}
\texttt{setenv FASTLIBS /seqdata/info/fastgbs} (csh/tcsh)\\
 or\\
\texttt{export FASTLIBS=/seqdata/info/fastgbs} (bash/ksh)
\end{quote}
 Then edit the \texttt{fastlibs} file to indicate the location of the
 protein and DNA sequence libraries.  If the protein sequence library is
 kept in the file \texttt{/seqdata/aa/swissprot.lseg} and your Genbank
 DNA sequence library is kept in the directory:
 \texttt{/seqdata/genbank}, then the \texttt{fastlibs} file might
 contain:
%%\pagebreak
\begin{verbatim}
SwissProt$0P/seqdata/aa/swissprot.lseg 0
UniProt$0+uniprot+@/seqdata/aa/uniprot.nam
GB Primate$1P@/seqdata/genbank/gpri.nam
GB Rodent$1R@/seqdata/genbank/grod.nam 
GB Mammal$1M@/seqdata/genbank/gmammal.nam
^   1    ^^^^       4                ^ ^
          23                           (5)
\end{verbatim}
The first line of this file says that there is a copy of the SwissProt
sequence database (a protein database) that can be selected by typing
"P" on the command line or when the database menu is presented in
interactive mode.

Note that there are 4 (or 5) fields in the lines in the
\texttt{fastlibs} file.  The first field describes library and is
displayed by FASTA program; it ends with the '\$'.  The second field
(1 character), is a 0 if the library is a protein library and 1 if it
is a DNA library.  The third field can either be a single character
(\texttt{P}) or a word surrounded by the \texttt{+} symbol
(\texttt{+uniprot+}), and can be used to specify the library on the command line or in interactive mode.

The fourth field is the name of the library file.  In the example
above, the \texttt{/seqdata/aa/swissprot.lseg} file contains the
entire protein sequence library.  Alternatively,
\texttt{/seqdata/aa/uniprot.nam} is a file of file names, which
contains a list of one or more library files.  Likewise, the DNA
library files are files of file names.

In addition, an optional fifth field can be used to specify the format
of the library file.  Alternatively, you can specify the library
format in a file of file names.  This field must be separated from the
file name by a space character ('\ ') from the filename.  FASTA can
read the libraries in the following formats:\\

\begin{tabular}{r l}
0 & FASTA (\texttt{>SEQID} - comment/sequence) \\
1 & Uncompressed Genbank (LOCUS/DEFINITION/ORIGIN)\\
2 & NBRF CODATA (ENTRY/SEQUENCE) (obsolete)\\
3 & EMBL/SWISS-PROT (ID/DE/SQ)\\
4 & Intelligenetics (;comment/SEQID/sequence) (obsolete)\\
5 & NBRF/PIR VMS (\texttt{>P1;SEQID}/comment/sequence) (obsolete)\\
6 & GCG (version 8.0) Unix Protein and DNA (compressed)\\
7 & FASTQ (sequence only, quality ignored)\\
9 & a script that is executed to produce FASTA format sequences \\ 
10 & subset format (</slib2/swissprot.lseg 0:2 4|) \\
11 & NCBI Blast1.3.2 format  (unix only) (obsolete)\\
12 & NCBI Blast2.0 format\\
16 & MySQL (requires special compilation) \\
17 & Postgres (requires special compilation) \\
\end{tabular}

Today, the most popular formats are \texttt{FASTA}, type \texttt{'0'},
the default, and the NCBI Blast2.0 \texttt{formatdb} formats (type
\texttt{'12'}).  The FASTA programs cannot read NCBI ASN.1 formatted databases.
If a library format is not specified, for example, because
you are just comparing two sequences, FASTA (format 0) is used by
default. To specify a library type on the command line, add it to the
library filename and surround the filename and library type in quotes:
\begin{quote}
\begin{verbatim}
fasta36 query.file "/seqdb/genbank/gbmam 12"
\end{verbatim}
\end{quote}
NCBI \texttt{formatdb} databases are built from multiple files,
e.g. \texttt{gbmam.nsq}, \texttt{gbmam.nhr}, \texttt{gbmam.nin}; to
refer to the complete set of files, simply use name before the
suffixes, e.g. \texttt{gbmam}.  When NCBI databases distributed across
several files, e.g. \texttt{gbbct.00}, \texttt{gbbct.01}, etc, those
files must be included in a \texttt{gbbct.nam} file of file names.

FASTA subset format ({\tt 10}) allows users to search a subset of a
sequence database, by specifying a list of {\tt gi} numbers or accessions in
a larger database.  The format begins with a line naming the file
sequence file followed by information about how to
extract the {\tt gi} number or accession. Thus, the line.
\begin{quote}
\texttt{<library\_file lib\_fmt:id\_fmt id\_loc}
\end{quote}
where {\tt lib\_fmt} is the library format (0), {\tt id\_fmt} is the
format of the sequence identifier (:1, :2 - ordered strings or
numbers; :3, :4 - unordered strings or numbers),
and {\tt id\_loc} is the location of the sequence identifier. For example,
\begin{quote}
\begin{verbatim}
</slib2/blast/swissprot.lseg 0:2 4|
3121763
51701705
7404340
74735515
...
\end{verbatim}
\end{quote}
specifies the file containing all the sequences and the file is in
FASTA format ('0:'), the sequence identifier is a number (':2'),
and the identifier starts at character 4 and ends with the \texttt{'|'} symbol. 

The major problem that most new users of the FASTA package have is in
setting up the program to find the databases and their library type.
In general, if you cannot get \texttt{fasta36} to read a sequence
database, there is probably something wrong with the \texttt{FASTLIBS}
file.  A common problem is that the database file is found, but either
no sequences are read, or an incorrect number of entries is read.
This is almost always because the library format (\texttt{libtype}) is
incorrect.

Test the setup by running FASTA.  Enter the sequence
file '\texttt{mgstm1.aa}' when the program requests it (this file is
included with the programs).  The program should then ask you to
select a protein sequence library.  Alternatively, if you run the
\texttt{tfastx36 -I} program and use the mgstm1.aa query sequence, the program
should show you a selection of DNA sequence libraries.
Once the \texttt{fastlibs} file has been set up correctly, you can
set FASTLIBS=fastgbs in your AUTOEXEC.BAT file, and you will not need to
remember where the libraries are kept or how they are named.

%%\pagebreak
\section{Frequently Asked Questions (FAQs)}

{\noindent}\textbf{Where can I get FASTA?} --

The most current version of the FASTA source code is available from
\url{https://github.com/wrpearson/fasta36}.  In addition, you can get
the programs from \url{https://fasta.bioch.virginia.edu/wrpearson/fasta},
but sometimes there is a lag between the latest release on GITHUB and
the compiled versions at \url{fasta.bioch.virginia.edu/wrpearson/fasta}.  This document
describes \texttt{\CURRENT}, which is available from
\url{https://fasta.bioch.virginia.edu/wrpearson/fasta/fasta3.tar.gz}.  In
addition, pre-compiled versions of the programs are available for
MacOSX and Windows.

\needspace{4\baselineskip}
{\noindent}\textbf{Which program should I use?} -- See Table \ref{table1}, also:\\

\begin{tabular}{l l l l l }
\hline \\[-1.0ex]
Query & Library & FASTA pgm. & BLAST pgm. & \\[1.2ex]
\hline \\[-1.0ex]
Prot. & Prot. & \texttt{fasta36} & \texttt{blastp} & heuristic local similarity \\
 &  & \texttt{ssearch36} &  & optimal local sim.\\
 &  & \texttt{ggearch36} &  & global:global sim. \\
 &  & \texttt{glearch36} &  & global:local sim.\\
DNA & DNA & \texttt{fasta36}$^*$ & \texttt{blastn} & \\[1.2ex]
\hline \\[-1.0ex]
Prot. & Prot. & \texttt{lalign36} & & multiple non-intersecting \\
DNA & DNA & & & alignments \\[1.2ex]
\hline \\[-1.0ex]
DNA & Prot. & \texttt{fastx36} & \texttt{blastx} & trans. DNA:protein sim. \\ 
 &  & \texttt{fasty36} & & \\[1.2ex]
\hline \\[-1.0ex]
Prot. & DNA & \texttt{tfastx36} & \texttt{blastn} & protein:trans. DNA \\
 &  & \texttt{tfasty36} & & \\[1.2ex]
\hline \\[-1.0ex]
Prot. & Prot. & \texttt{fasts36} & & Unordered peptides \\
Prot. & DNA & \texttt{tfasts36} & & Unordered peptides \\
DNA & DNA & \texttt{fasts36} & & Unordered oligonucleotides \\
Prot. & Prot. & \texttt{fastm36} & & Ordered peptides \\
DNA & DNA & \texttt{fastm36} & & Ordered oligos \\[1.2 ex]
\hline \\[-1.0ex]
\multicolumn{5}{l}{$^*$\texttt{ssearch36} can also be used for DNA:DNA, but is much slower and no more sensitive.}\\[0.2ex]
\hline \\
\end{tabular}

\needspace{4 ex}
{\noindent}\textbf{How do I make FASTA act/look like BLAST}? --
\vspace{-0.5ex}
\begin{quote}
\texttt{fasta36 -s BP62 -m BB query.file library.file}
\end{quote}
\vspace{-0.5ex}
\texttt{-s BP62} sets the same scoring matrix (BLOSUM62) and
gap-penalties (-11/-1) as BLAST (FASTA uses BLOSUM50 by
default). \texttt{-m BB} produces very BLAST-like output.

In addition, the \texttt{-m 8} and \texttt{-m 8C} options provide
BLAST tabular output, optionally with comments (\texttt{-m 8C}). This
compact output is effective for analysis pipelines.  In addtion,
\texttt{-m 8XC} (no comments) or \texttt{-m 8CC} provides two
additional blast-tabular fields, a CIGAR alignment string and, if
available, an annotation string.

{\noindent}\textbf{When I search Genbank - the program reports:} \texttt{0 residues in 0
sequences}?  This typically happens because the program does not
know that you are searching a Genbank flatfile database and is looking
for a FASTA format database.  Be certain to specify the library type
("1" for Genbank flatfile) with the database name.

{\noindent}\textbf{The search seemed to work, but I do not see any results.} -- In
command line mode (the default), all the FASTA programs limit the
number of high scoring sequences shown using an expectation value
cutoff ($E()<10$ for proteins; $E()<2$ for DNA).  Sometimes, a search
will complete successfully (you see the message \texttt{XXXX residues
  in YYY sequences}) but the message: \texttt{!! No sequences with E()
  < 10} instead of \texttt{The best scores are:}.  Typically, this
happens because of a problem with the statistical estimation process;
in particular, if the library contains only related sequences and
\texttt{-z 11} was not used, none of the hits may be ``significant''.
To trouble shoot this problem, you can search with \texttt{-z -1},
which turns off all the statistical estimation procedures, and will
show the 20 highest scoring sequences (\texttt{-b \#} sets the default
number of sequences shown).

{\noindent}\textbf{What is the difference between} \texttt{fastx3} and
\texttt{fasty3}? (or \texttt{tfastx3} and \texttt{tfasty3})? --
\texttt{[t]fastx3} uses a simpler codon based model for alignments
that does not allow frameshifts in some codon positions (see
ref. \cite{wrp971}).  \texttt{fastx3} is about 30\% faster, but
\texttt{fasty3} can produce higher quality alignments in some cases.

\vspace{0.5ex}
{\noindent}\textbf{What is ktup}? -- All of the programs with \texttt{fast} in their
name use a computer science method called a lookup table to speed the
search.  For proteins with \emph{ktup}=2, this means that the program
does not look at any sequence alignment that does not involve matching
two identical residues in both sequences.  Likewise with DNA and
\emph{ktup} = 6, the initial alignment of the sequences looks for 6
identical adjacent nucleotides in both sequences.  Because it is less
likely that two identical amino-acids will line up by chance in two
unrelated proteins, this speeds up the comparison.  But very distantly
related sequences may never have two identical residues in a row but
will have single aligned identities.  In this case, \emph{ktup} = 1 may
find alignments that \emph{ktup}=2 misses.

\vspace{0.5ex} {\noindent}\textbf{How do I turn off statistics}? --
The FASTA programs are designed to identify homologs based on
statistically significant similarity; to infer homology you need
accurate statistical estimates.  Sometimes, however, you know the
sequences are related, and searching against libraries of related
sequences can confuse FASTA if you do not use \texttt{-z 11}. If all
you want are scores and alignments, use \texttt{-z -1} to turn off
statistical estimates.

\vspace{0.5ex}
{\noindent}\textbf{Where are} \texttt{prss} {\noindent}\textbf{and} \texttt{prfx}? -- Earlier FASTA3
releases included \texttt{prss3} and \texttt{prfx3}. With FASTA
version 35 and 36, these programs have been incorporated into
\texttt{ssearch36} and \texttt{fastx36}.  FASTA version 35 and 36
programs now automatically estimate statistical parameters by
shuffling - the function of \texttt{prss} and \texttt{prfx}, when
searching for libraries with fewer than 500 members.

\vspace{0.5ex}
{\noindent}\textbf{Where is} \texttt{tfasta}? -- Although it is possible to make
\texttt{tfasta36}, it is not compiled by default.  \texttt{tfastx36}
and \texttt{tfasty36} allow frame-shifts to be joined into a single
alignment; \texttt{tfasta} did not. \texttt{tfastx36} produces better
alignments with better statistics.

\vspace{0.5ex}
{\noindent}\textbf{Can I run the FASTA programs on a cluster}? -- With version
36.3.4, almost all of the FASTA programs can be run on clusters of
computers using MPI (Message Passaging Interface).  The programs can
be compiled using \texttt{make -f ../make/Makefile.mpi\_sse2} from the
\texttt{fasta36/src} directory.  Except for \texttt{lalign36}, all the
programs in Table \ref{table1} are available as \texttt{fasta36\_mpi},
\texttt{ssearch36\_mpi}, etc.

Unfortunately, the current MPI implementation involves substantially
more communications overhead than the threaded versions.  The FASTA
programs are very efficient on threaded machines; if the preload
option is used (edit \texttt{make/Makefile36m.common} to use
\texttt{comp\_lib8.c}), the FASTA programs can obtain more than 40-fold speedup on a 48-core machine (the largest I have tested). 

\vspace{0.5ex}
{\noindent}\textbf{Sometimes, in the list of best scores, the same sequence is
  shown twice with exactly the same score.  Sometimes, the sequence is
  there twice, but the scores are slightly different}? -- When any of
the FASTA programs searches a long sequence, it breaks the sequence up
into \emph{overlapping} pieces.  If the highest scoring alignment is
at the end of one piece, it will be scored again at the beginning of
the next piece.  If the alignment is not be completely included in the
overlap region, one of the pieces will give a higher score than the
other.  These duplications can be detected by looking at the
coordinates of the alignment.  If either the beginning or end
coordinate is identical in two alignments, the alignments are at least
partially duplicates.

\vspace{2ex}
As always, please inform me of bugs as soon as possible.

\begin{quote}
William R. Pearson\\
Department of Biochemistry\\
Pinn Hall Box 800733\\
U. of Virginia\\
Charlottesville, VA\\
wrp@virginia.EDU
\end{quote}

\bibliographystyle{plain}
\bibliography{fasta_guide}

\appendix
\section*{Appendix}

\section{FASTA Makefile compile time options}

\begin{table}
\caption{\label{make-defs}FASTA \texttt{Makefile} compile time \texttt{\#defines}}
\vspace{1.0ex}
\begin{tabular}{l l p{1.00 in} p{3.0 in}}
\hline\\[-1.2ex]
\texttt{\#define} & Status$^*$ & Target file(s) & Function \\[1.0ex]
\hline\\[-1.5ex]
\texttt{ALLOCN0} & obs & \texttt{dropnfa.c}, \texttt{dropfx.c}, \texttt{dropfz2.c} & allows FASTA algorithm to use memory $\sim$ query length (n0), not query $+$ library (n0+n1). \\
\texttt{DNALIB\_LC} & undef & \texttt{initfa.c} & enable lower case masking for DNA libraries \\
\texttt{HTML\_HEAD} & undef & \texttt{comp\_lib5e.c}, \texttt{comp\_lib8.c} & wrap \texttt{-m 6} HTML output with \texttt{<html> <body> </body> </html>} \\
\texttt{M10\_CONS} & def & \texttt{c\_dispn.c} & show consensus line (\texttt{:. }) with \texttt{-m 10} output. \\
\texttt{OLD\_FASTA\_GAP} & undef & \texttt{drop*.c} & use first-residue/additional residue penalties, not open/extend. \\
\texttt{PGM\_DOC} & def & \texttt{comp\_lib5e.c}, \texttt{comp\_lib8.c}  & provide \texttt{\#pgm\_name -opt1 -opt2 query file} copy of command line \\
\texttt{PROGRESS} & def & \texttt{comp\_lib5e.c}, \texttt{comp\_lib8.c}  & provide progress symbols in interactive mode \\
\texttt{SAMP\_STATS} & def & \texttt{comp\_lib5e.c}, \texttt{comp\_lib8.c}  & scores are sampled for statistical estimates \\
\texttt{SAMP\_STATS\_LESS} & def & \texttt{compacc.c} & a slower sampling strategy is used \\
\texttt{SHOW\_ALIGN\_SCORE} & undef & \texttt{wm\_align.c} & print score, cummulative score, during alignment (for teaching) \\
\texttt{SHOW\_HELP} & def & \texttt{comp\_lib5e.c}, \texttt{comp\_lib8.c}, \texttt{initfa.c}, \texttt{doinit.c} & print out help information with '-help', or no arguments given.  Undef \texttt{SHOW\_HELP} reverts to pre-\texttt{fasta-35.4.4}.\\
\texttt{SHOW\_HIST} & undef & \texttt{doinit.c} & inverts current meaning of \texttt{-H} (shows by default for non-PCOMPLIB (MPI) programs). \\
\texttt{SHOWSIM} & def & \texttt{mshowbest.c} \texttt{mshowalign2.c} & display percent similarity \\
\texttt{USE\_LNSTATS} & obs & \texttt{scaleswn.c} & use $ln()$-scaling for scores, removed in \texttt{fasta2.0}.\\
\hline \\
\end{tabular}
$^*$Status: def: \#defined in standard \texttt{Makefiles}; undef: undefined; obs: obsolete, provided backwards compatibility with FASTA2.0 or earlier.
\end{table}

The \texttt{fasta-36/make} directory includes \texttt{Makefile}s
appropriate for a broad range of environments, including Linux/Unix,
BSD, MacOSX, and Windows.  Makefiles are regularly tested against
MacOSX, Linux, and Windows. Table \ref{make-defs} summarizes the
Makefile options that can be modified.

As distributed, the \texttt{Makefiles} in \texttt{fasta36/make}, build
a version of the FASTA programs that is optimized for single searches
against arbitrary sized databases, using bit scores, efficient sampled
statistics, and gap-open/extend penalties.  The default compilation
configuration can be changed either by changing the compile time
defines (Table \ref{make-defs}) in the main \texttt{Makefile},
e.g. \texttt{make/Makefile.linux64\_sse2}, or by editing
\texttt{make/Makefile36m.common}.

\emph{High-performance searches with many queries} -- By default, the
\texttt{comp\_lib5e.c} program specified in
\texttt{Makefile36m.common} builds FASTA programs that re-read the
library sequence database for every query sequence.  This has the
advantage that sequence comparison begins almost immediately, but if
thousands of searches are being performed, the database is re-read
thousands of times.  \texttt{Makefile36m.common} can be edited to use
\texttt{comp\_lib8.c} in place of \texttt{comp\_lib5e.c} and the
database is read only once, then held in memory for additional
searches.  Of course, if \texttt{comp\_lib8.c} is used, the computer
must have enough memory to store the complete database.  Keeping the
database in memory allows the FASTA programs to very efficiently used
large, multicore computers.

\needspace{5\baselineskip}
\emph{Parallel searches with MPI} -- By default under
Unix/Linux/MacOSX, the FASTA programs are threaded; they will spawn as
many threads as CPU cores are available (this can be limited with the
\texttt{-t n-threads} option).  Using \texttt{comp\_lib8.c}, we see
almost 48-fold speedup on a 48-core machine.  The FASTA programs can
also be run in parallel in the MPI environment on clusters of
computers.  To build the MPI versions of the programs, use
\texttt{make ../make/Makefile.mpi\_sse2 ssearch36\_mpi},
\texttt{fastx36\_mpi}, etc.  The MPI programs currently substantially
more communications overhead than the threaded versions, so they may
not scale as well to large clusters.

\begin{longtable}{p{0.75 in}p{5.25 in}}
\multicolumn{2}{c}{\textbf{FASTA version history (cont.)}} \\
\hline\\[-1.0ex]
% \textbf{Date} & {\bf Improvements} \\[0.5ex] \hline \\[-1.5ex]
\endhead
\multicolumn{2}{l}{{\Large {\bf B FASTA version history}}} \\[2 ex]
\hline\\[-1.0ex]
% {\bf Date} & {\bf Improvements} \\[0.5ex] \hline \\[-1.5ex]
\endfirsthead
\hline\\
& \\
\endfoot
\hline\\
& \\
\endlastfoot

\multicolumn{2}{c}{ \FASTA v33, Oct, 1999 -- Dec, 2000 } \\[1 ex]
\hline \\[-0.5 ex]

Oct 1999 & Add support for NCBI Blast2.0 formatted libraries, and
memory mapped databases.  \FASTA now reads both \texttt{BLAST1.4} and
\texttt{BLAST2.0} formatted databases. (version 3.2t08)\\ & Include
Maximum Likelihood Estimates for Lambda and K ( -z 2) \\

 & Include a new strategy for searching with low
complexity regions.  The \texttt{pseg} program can produce libraries
with low complexity regions as lower case characters, which can be
ignored during the initial \texttt{FASTA}/\texttt{SSEARCH} scan, but are considered when
producing the final alignments. (3.3t01)\\

 & Change output to report bit scores, which are also  used by BLAST. \\

Mar 2000 & Another new statistics option, -z 6, uses Mott's
approach \cite{mot921} for calculating a
composition dependent Lambda for each sequence. (3.3t05) \\

Dec 2000 & Automatically change the gap penalties when alternate
(known) scoring matrices are used using Reese and Pearson gap
penalties \cite{wrp022}. First implementation to read from MySQL
databases. \\ May 2001 & change all \FASTA gap penalties from
first-residue, additional residue to the gap-open, gap-extend values
used by BLAST. \\[0.5ex]

\hline \\[-0.5 ex]
\multicolumn{2}{c}{ \FASTA v34, Jan, 2001 -- Jan, 2007 } \\[1 ex]
\hline \\[-0.5 ex]

Jun 2002 & Modify statistical estimation strategy to sample all the
sequences in the database, not just the first 60,000. (3.4t11) \\

Jan 2003 & Implementation of vector-accelerated (Altivec) code for
Smith-Waterman ({\tt SSEARCH}) and banded Smith-Waterman (\FASTA)
using the Rognes and Seebug \cite{rog003} algorithm.  This code was
removed in Sept, 2003, because of possible conflict with a patent
application, but was restored using a different algorithm in
Nov. 2004. \\

Jun 2003 & Provide \texttt{PSI-SEARCH} --- an implementation of
\texttt{SSEARCH} that can search with \texttt{PSI-BLAST} PSSM profile
files.  \texttt{PSI-SEARCH} estimates statistical significance from
the distribution of actual alignment scores; thus the estimates are
much more reliable than \texttt{PSI-BLAST} estimates.  Also, change
the similarity display to work with profiles. (3.4t22) \\

July 2003 & Provide ASN.1 definition line parsing for \texttt{BLAST}
{\tt formatdb} v.4 libraries.  Restructure the programs to use a table-driven
approach to parameter setting.   Two tables now define the algorithm,
query sequence type, library type, scoring matrix, and gap penalties for
all programs.  \\

Sept 2003 & A new option {\tt -V} for annotating alignments
provided. Designed for highlighting post-translational modifications
with {\tt fasts}, it can also be used to highlight active sites and
other conserved residues. (3.4t23) \\

Dec 2003 & Addition of {\tt -U} option for RNA sequence
comparison. {\tt G:A} matches score like {\tt G:G} matches to account
for {\tt G:U} basepairs.  Change default {\it ktup} for short query
sequences.  Increase band-width for DNA banded final alignments. \\

July 2004 & Allow searching of \texttt{Postgres}, as well as
\texttt{MySQL} database queries. \\

Nov 2004 & (\texttt{fa34t24}) Incorporation of Erik Lindahl "anti-diagonal" Altivec
implementation of \cite{woz974} for Smith-Waterman only.  Altivec
{\tt ssearch34} is now faster than {\tt fasta34} for query sequences $<$ 250 amino acids. \\

Jan 2005 & Change {\tt FASTS} to accommodate very large numbers of
peptides ($>$100) for full coverage on long proteins \\

Jun. 2006 & (\texttt{fa34t26}) Incorporation of Smith-Waterman
algorithm for the SSE2 vector instructions written by Michael Farrar
\cite{farrar2007}.  The SSE code speeds up Smith-Waterman 8 --
16-fold. \\[1.0 ex]

\hline \\[-0.5 ex]
\multicolumn{2}{c}{ \FASTA v35, March, 2007 -- March, 2010 } \\[1 ex]
\hline \\[-0.5 ex]

Mar. 2007 & fasta v35 -- Accurate shuffle-based $E()$-values for all searches and alignments; statistics from searches against small libraries are supplemented with shuffled alignments.\\[1 ex]

 & More efficient threading strategies on multi-core computers, for 12X speedup on 16-core machines.\\[1 ex]

 & Inclusion of \texttt{lalign} (\texttt{SIM}) local domain alignments. \texttt{lalign} alignments now have accurate shuffle-based $E()$-values.\\[1 ex]

Apr. 2007 & Introduction of \texttt{ggsearch}, for global alignment searches, and \texttt{glsearch}, for searches with scores that are global in the query and local in the library.  \texttt{ggsearch} and \texttt{glsearch} calculate $E()$-values using the normal distribution.  Both programs can search with \texttt{PSI-BLAST} PSSMs.\\[1 ex]

Dec. 2007 & Efficient strategy for searching subsets of databases (lists of GI or accession numbers) \\[1 ex]

Feb. 2008 & Annotations in either query or library sequences can be highlighted in the alignment, and the state of annotated residues is compactly summarized with \texttt{-m 9c}. \\[1 ex]

Oct. 2008 & Modification \texttt{lsim4.c} (\texttt{lalign35}) provided by Xiaoqui Huang to ensure
that self-alignments do not cross the identity diagonal. \\[1ex]
%\pagebreak
\hline \\[-0.5 ex]
\multicolumn{2}{c}{ \FASTA v36, March, 2010 -- } \\[1 ex]
\hline \\[-0.5 ex]

Mar. 2010 & \FASTA v36 displays all significant alignments between
query and library sequence.  BLAST has always displayed multiple
high-scoring alignments (HSPs) between the query and library sequence;
previous versions of the FASTA programs displayed only the best
alignment, even when other high-scoring alignments were present.\\[1
  ex]

&  New statistical options, \texttt{-z 21, 22, 26}, provide a second $E2()$-value
estimate based on shuffles of the highest scoring sequences. \\[1 ex]

  &  Improved performance using statistics-based thresholds for
  gap-joining and band-optimization in the heuristic FASTA local
  alignment programs, increasing speed 2 - 3X. \\[1 ex]

  &  Greater flexibility in specifying combinations of library files
  and subsets of libraries.  \FASTA v36
  programs can include indirect files of library names inside of
  indirect files of library names. \\[1 ex]

  & \FASTA36 programs are fully threaded, both for
  searches, and for alignments.  The programs routinely run 12 - 15X
  faster on 8-core machines with "hyperthreading" (effectively 16 cores).
  \\[1 ex]

 & \texttt{-z 21} .. \texttt{26} E2() statistical estimates from
  shuffled best scores.\\[1.0ex]

Sep. 2010 & \texttt{-m 8}, \texttt{-m 8C} BLAST tabular output. \\[1.0ex]

Nov, 2010 & Variable scoring matrices (\texttt{-m ?BP62}).\\[1.0ex]

Dec, 2010 & (\texttt{fasta-36.3.1}) SSE2 vectorized \texttt{ggsearch36}, \texttt{glsearch36} (Michael Farrar).\\[1.0ex]

Jan, 2011 & (\texttt{fasta-36.3.2}) MPI versions implemented and tested.\\[1ex]

Feb, 2011 & Introduce \texttt{-m B}, \texttt{-m BB} BLAST-like output.\\[1.0ex]

Mar, 2011 & (\texttt{fasta-36.3.4}) Program is no longer interactive by
default. \texttt{fasta36 -h} and \texttt{fasta36 -help} provide
common/complete options, with many defaults. \texttt{doc/fasta\_guide.pdf} available.\\[1.0ex]

May, 2011 & (\texttt{fasta-36.3.5}) Introduce (1) \texttt{-e
  expand.sh} scripts to extend the effective size of the database
searched, based on significant hits; (2) \texttt{-m "F\# output.file"}
to send different output formats to different files; and (3)
\texttt{-X} expanded options, \texttt{-o} replaces the old \texttt{-X}
and \texttt{-Xo} replaces \texttt{-o}. \\[1.0ex]

Jan, 2012 & Include \texttt{.fastq} files as library type 7 \\[1.0ex]

May, 2012 & allow reverse-complement alignments with \texttt{ggsearch} and \texttt{glsearch} \\[1.0ex]

Jun, 2012 & Introduce \texttt{-V !script.pl} driven alignments, and variant scoring.\\[1.0ex]

Aug, 2012 & Introduce \texttt{-V !ann\_feats.pl} sub-alignment (region-based) scoring.\\[1.0ex]

Apr, 2013 & Extend \texttt{ENV} options to introduce a domain-plotting option for FASTA web sites.\\[1.0ex]

\hline
\end{longtable}

\end{document}
